% This is part of the book TeX for the Impatient.
% Copyright (C) 2003 Paul W. Abrahams, Kathryn A. Hargreaves, Karl Berry.
% Copyright (C) 2004 Marc Chaudemanche pour la traduction fran�aise.
% See file fdl.tex for copying conditions.

\input fmacros
\chapter{Comprendre les \linebreak messages d'erreur}

\chapterdef{errors}

%\codefuzz = 4pc % for this chapter only (in scope of \chapter)
\bix^^{messages d'erreur}
L'interpr\'etation des messages d'erreur de \TeX\ peut parfois \^etre 
compar\'ee \`a aller chez votre m\'edecin avec la sensation d'\^etre 
fatigu\'e et obtenir, en r\'eponse, une analyse de votre chimie sanguine. 
C'est probablement l'explication de votre d\'etresse, mais il n'est pas 
facile de la d\'ecrire. Quelques r\`egles simples vous feront aller plus 
loin en vous aidant \`a comprendre les messages d'erreur de \TeX\ et \`a en 
obtenir d'avantage.

Votre premier objectif sera de comprendre ce que vous avez fait pour que 
\TeX\ se plaigne. Le second (si vous travaillez interactivement) sera de 
d\'eceler le plus d'erreurs que vous pouvez en une seule fois. 

Regardons un exemple. Supposez que votre source contienne la ligne~: 
\csdisplay 
We skip \quid a little bit. 
| 
Vous pensiez saisir `|\quad|', mais vous avez tap\'e `|\quid|' \`a la 
place. Voici la r\'eponse que vous obtiendrez de \TeX\~: 
\csdisplay !! 
Undefined control sequence. 
l.291 We skip \quid 
                    a little bit. 
? 
| 
Ce message appara\^\i tra sur votre terminal et dans votre \logfile. 
^^{fichier log//message d'erreur} La premi\`ere ligne, qui commence 
toujours par un point d'exclamation (|!!|), vous indique ce qu'est le 
probl\`eme. Les deux derni\`eres lignes avant le prompt `|?|'(qui sont, 
dans ce cas-ci, \'egalement, les deux lignes suivantes) vous indiquent o\`u 
\'etait \TeX\ quand il a trouv\'e l'erreur. Il a trouv\'e l'erreur sur la 
ligne $291$ du fichier source courant, et la coupure entre les deux lignes 
de message indiquent la position pr\'ecise de \TeX\ dans la ligne $291$, \`a 
savoir, juste apr\`es |\quid|. Le fichier source courant est celui juste 
apr\`es la parenth\`ese gauche la plus r\'ecemment ouverte dans le r\'esultat 
de votre ex\'ecution sur votre terminal (voir les \xref{infiles}).

Cette erreur particuli\`ere, un ``undefined control sequence'', est l'une 
des plus communes que vous puissiez obtenir. Si vous r\'epondez au prompt 
par un autre~`|?| ', \TeX\ affichera le message suivant~:

{\hfuzz = 2in
\csdisplay
Type <return> to proceed, S to scroll future error messages,
R to run without stopping, Q to run quietly,
I to insert something, E to edit your file,
1 or ... or 9 to ignore the next 1 to 9 tokens of input,
H for help, X to quit.
|
}%
Voici ce que signifient ces propositions~:
\ulist
\li Si vous saisissez \asciichar{return}, \TeX\ continuera \`a traiter 
votre document. Dans ce cas, il ignorera juste le |\quid|.
\li Si vous saisissez `|S|' (ou `|s|'---majuscule et minuscule sont 
\'equivalentes ici), \TeX\ traitera votre document sans s'arr\^eter, 
\emph{sauf} s'il rencontre un fichier absent. Les messages d'erreur 
continueront \`a appara\^\i tre sur votre terminal et dans le \logfile.
\li Si vous saisissez `|R|' ou `|r|', vous obtiendrez le m\^eme effet que 
pour `|S|', sauf que \TeX\ ne s'arr\^etera pas pour les fichiers manquants.
\li Si vous saisissez `|Q|' ou `|q|', \TeX\ continuera \`a traiter votre 
document mais ne s'arr\^etera pas pour des erreurs ni ne les montrera sur 
votre terminal. Les erreurs appara\^\i tront toujours dans le \logfile.
\li Si vous saisissez `|X|' ou `|x|', \TeX\ nettoiera ce qu'il peut au 
mieux, jettera la page sur laquelle il travaillait et stoppera. Vous pourrez 
imprimer ou regarder les pages que \TeX\ a d\'ej\`a trait\'ees. 
\li Si vous saisissez `|E|' ou `|e|', \TeX\ nettoiera et se terminera comme 
il le fait pour `|X|' ou  `|x|' et entrera alors dans votre \'editeur de 
texte, vous pla\c cant sur la ligne incorrecte. (tous les syst\`emes ne 
supportent pas cette option.)
\li Si vous saisissez `|H|' ou `|h|', vous obtiendrez une autre explication 
de l'erreur montr\'ee sur votre terminal et probablement du conseil sur 
quoi faire \`a son sujet. Cette explication appara\^\i tra \'egalement dans 
votre \logfile. Pour le ``undefined control sequence'' ci-dessus, vous 
obtiendrez~:

{\hfuzz = 2in
\csdisplay
The control sequence at the end of the top line
of your error message was never \def'ed. If you have
misspelled it (e.g., `\hobx'), type `I' and the correct
spelling (e.g., `I\hbox'). Otherwise just continue,
and I'll forget about whatever was undefined.
|
}
\li Si vous saisissez `|?|', vous recevrez encore ce m\^eme message.
\endulist
\noindent
Les deux autres solutions de rechange, saisir `|I|' ou un petit nombre 
entier, fournissent des moyen de demander \`a \TeX de revenir en \ arri\`ere 
pour que votre erreur ne cause pas d'autres erreurs plus tard dans 
votre document~:

\ulist
\li Si vous saisissez `|I|' ou `|i|' suivi de texte, alors \TeX\ ins\'erera 
ce texte comme s'il s'\'etait produit juste apr\`es le point d'erreur, au 
niveau les plus secrets o\`u \TeX \ travaille. Dans le cas de l'exemple 
ci-dessus, cela signifie \`a la position de \TeX\ dans votre source 
original, c'est-\`a-dire, juste apr\`es `|\quid|'. Plus tard vous verrez un 
exemple qui montre la diff\'erence entre ins\'erer quelque chose au niveau 
les plus secrets et l'ins\'erer dans votre source original. Dans l'exemple 
ci-dessus d'``undefined control sequence'', si vous saisissez~:
\csdisplay
I\quad
|
\TeX\ effectuera la commande |\quad| et produira un espace cadratin 
l\`a o\`u vous avez eu l'intention d'avoir un.
\li Si vous saisissez un nombre entier positif inf\'erieur \`a $100$ (pas 
inf\'erieur \`a $10$ comme le message le sugg\`ere maladroitement), \TeX\ 
supprimera ce nombre de tokens du niveau les plus secrets o\`u il 
travaille. (si vous saisissez un nombre entier sup\'erieur ou \'egal \`a 
$100$, \TeX\ supprimera $10$ tokens~!)
\endulist

Voici un autre exemple d'erreur commune~: 
\csdisplay
Skip across \hskip 3cn by 3 centimeters.
|
Le message d'erreur pour ceci est~:
\csdisplay
!! Illegal unit of measure (pt inserted).
<to be read again> 
                   c
<to be read again> 
                   n
l.340 Skip across \hskip 3cn
                             by 3 centimeters.
|
^^|<to be read again>|
Dans ce cas, \TeX\ a vu que `|3|' est suivi par quelque chose qui n'est pas 
une unit\'e de mesure reconnue, et ainsi suppose que l'unit\'e de mesure est 
le point. \TeX\ relira le token `|cn|' et l'ins\'erera dans votre saisie, 
ce qui n'est pas ce que vous voulez. Dans ce cas, vous pouvez obtenir un 
meilleur r\'esultat en saisissant d'abord `|2|' pour reculer avant `|cn|'. 
Vous recevrez le message~:
\csdisplay
<recently read> n
                 
l.340 Skip across \hskip 3cn
                           by 3 centimeters.
|

Maintenant vous pouvez saisir `|I\hskip 3cm|' pour obtenir le saut que vous 
vouliez (en plus du saut de |3pt| que vous avez d\'ej\`a obtenu).\footnote{en 
saisissant `|I\unskip\hskip 3cm|' vous pouvez vous d\'ebarrasser du saut de 
|3pt|.}

Si vous saisissez quelque chose qui n'est valide qu'en mode math\'ema\-tique, 
\TeX\ s'orientera vers le mode math\'ematique pour vous que ce soit ce que 
vous avez vraiment voulu ou non. Par exemple~:
\csdisplay
So \spadesuit s are trumps.
|
Voici le message d'erreur de \TeX\~:
\csdisplay
!! Missing $ inserted.
<inserted text> 
                $
<to be read again> 
                   \spadesuit 
l.330 So \spadesuit
                  s are trumps.
|
Puisque le symbole |\spadesuit| n'est permis qu'en mode math\'ematique, 
\TeX\ a ins\'er\'e un `|$|' devant lui. Apr\`es que \TeX\ ait ins\'er\'e un 
token, il se place \emph{devant} ce token, dans ce cas, le `|$|', pr\^et 
\`a le lire. saisir `|2|' fera que \TeX\ sautera le `|$|' et le token 
`|\spadesuit|', le laissant pr\^et \`a traiter le `|s|' de 
`|s are trumps|'. Si vous laissez juste \TeX\ continuer, il 
composera `|s are trumps|' en mode math\'ematique.

Voici un exemple o\`u le diagnostic d'erreur de \TeX\ est compl\`etement 
erron\'e~:
\csdisplay
\hbox{One \vskip 1in two.}
|
Le  message d'erreur est~:
\csdisplay
!! Missing } inserted.
<inserted text> 
                }
<to be read again> 
                   \vskip 
l.29 \hbox{One \vskip
                     1in two.}
|
^^|<inserted text>|
Le probl\`eme est que vous ne pouvez pas utiliser |\vskip| quand \TeX\ est 
en mode horizontal restreint, c'est-\`a-dire, en construisant un hbox. Mais 
au lieu de rejeter le |\vskip|, \TeX\ a ins\'er\'e une accolade droite devant 
lui afin d'essayer de fermer le hbox. Si vous acceptez la correction de \TeX, 
\TeX\ se plaindra encore quand il arrivera plus tard \`a l'accolade droite 
correcte. Il se plaindra \'egalement \`a propos de tout ce qui pr\'ec\`ede cette
accolade droite et qui n'est pas permis en mode vertical. Ces plaintes 
r\'ep\'etitives seront particuli\`erement embrouillantes parce que les 
erreurs qu'elles indiquent sont fausses, elles sont le r\'esultat des effets 
propag\'es de l'insertion inad\'equate de l'accolade droite. Votre meilleur 
pari est de saisir `|5|', et de sauter apr\`es tous les tokens de `|}\vskip 1in|'.

Voici un exemple similaire dans lequel le message d'erreur est plus long 
que ce que nous avons vus jusqu'ici~:
\csdisplay
\leftline{Skip \smallskip a little further.} But no more.
|
L'erreur ici est que |\smallskip| ne fonctionne qu'en mode vertical. 
Le message d'erreur est quelque chose comme~:
\csdisplay
!! Missing } inserted.
<inserted text> 
                }
<to be read again> 
                   \vskip 
\smallskip ->\vskip 
                    \smallskipamount 
<argument> Skip \smallskip 
                           a little further.
\leftline #1->\line {#1
                       \hss }
l.93 ...Skip \smallskip a little further.}
                                           But no more.
|
Les messages d'erreur ici vous donnent un tour d'horizon des macros qui sont 
utilis\'ees dans l'impl\'ementation de |\leftline| dans \plainTeX---des 
macros sur lesquelles vous ne vous inqui\'etez probablement pas. La premi\`ere 
ligne vous indique que \TeX\ pr\'evoit de traiter le probl\`eme en 
ins\'erant une accolade droite. \TeX\ n'a pas encore r\'eellement lu 
l'accolade droite, ainsi vous pouvez la supprimer si vous le choisissez. 
Chaque composant du message apr\`es la premi\`ere ligne (celle avec le `!') 
occupe une paire de lignes. Voici ce que signifient les paires de lignes 
successives~:
\olist
\li La premi\`ere paire indique que \TeX\ a ins\'er\'e, mais pas encore lu, une 
accolade droite~:
\li La paire suivante indique qu'apr\`es avoir lu l'accolade droite, \TeX\ 
relira un `|\vskip|' (obtenue de la macro d\'efinition du |\smallskip|).
\li La troisi\`eme paire indique que \TeX\ d\'eveloppait la macro 
|\smallskip| quand il a trouv\'e l'erreur. Ces deux lignes  montrent 
\'egalement la d\'efinition de |\smallskip| et indiquent o\`u \TeX\ est 
parvenu en d\'evelop\-pant et en ex\'ecutant cette d\'efinition. 
Sp\'ecifiquement, il a juste essay\'e sans succ\`es d'ex\'ecuter la 
commande |\vskip|. En g\'en\'eral, une ligne diagnostic qui commence par 
une commande suivie de `|->|' indique que \TeX\ a d\'evelopp\'e et 
ex\'ecut\'e une macro de ce nom.
\li La quatri\`eme paire indique que \TeX\ traitait un argument de macro 
quand il a trouv\'e le |\smallskip| et indique \'egalement la position de 
\TeX\ du fait de l'argument, c'est-\`a-dire, qu'il a juste trait\'e le 
|\smallskip| (sans succ\`es). En regardant vers la prochaine paire de 
lignes nous pouvons voir que l'argument a \'et\'e pass\'e \`a |\leftline|.
\li La cinqui\`eme paire indique que \TeX\ d\'eveloppait la macro  
|\leftline| quand il a trouv\'e l'erreur. Dans cet exemple l'erreur s'est 
produite pendant que \TeX\ \'etait en train d'interpr\'eter plusieurs 
d\'efinitions de macro \`a diff\'erents degr\'es de d\'eveloppement. Sa 
position apr\`es |#1| indique que la derni\`ere chose qu'il ait vu \'etait le 
premier (et dans ce cas-ci le seul) argument de |\leftline|.
\li La derni\`ere paire indique o\`u \TeX\ est plac\'e dans votre fichier 
source. Notez que cette position est bien au del\`a de la position o\`u il 
a ins\`er\'e l'accolade droite et relu le `|\vskip|'. C'est parce que \TeX\ 
a d\'ej\`a lu tout l'argument du |\leftline| de votre fichier source, 
m\^eme s'il n'a trait\'e qu'une partie de cet argument. Les points au 
d\'ebut de la paire indiquent qu'une partie pr\'ec\'edente de la ligne 
source n'est pas montr\'ee. Cette partie pr\'ec\'edente, en fait, inclut la 
commande |\leftline| qui a rendu le |\vskip| ill\'egal.
\endolist
\noindent
Dans un long message comme ceci, vous ne trouverez g\'en\'eralement utiles 
que la premi\`ere ligne et la derni\`ere paire de lignes~; mais cela aide 
parfois de savoir ce que sont les autres lignes. N'importe quel texte que 
vous ins\'ererez ou effacerez sera ins\'er\'e ou supprim\'e au niveau les 
plus secrets. Dans cet exemple l'insertion ou la suppression se produira 
juste avant l'accolade droite ins\'er\'ee. Notez en particulier que dans ce 
cas-ci, \TeX \ met tout texte que vous pourriez ins\'erer, \emph{non} dans 
votre texte source mais dans une d\'efinition de macro plusieurs niveaux 
plus bas. (naturellement, la d\'efinition de macro originale n'est pas 
modifi\'ee.)

Vous pouvez employer la commande ^|\errorcontextlines| 
\ctsref{\errorcontextlines} pour limiter le nombre de paires de lignes de 
contexte d'erreur que \TeX\ produit. Si vous n'\^etes pas int\'eress\'e par 
toute l'information que \TeX\ vous fournit, vous pouvez placer |\error!-
contextlines| \`a $0$. Cela vous ne donnera que les premi\`eres et 
derni\`eres paires de lignes.

En conclusion, nous mentionnerons deux autres indicateurs qui peuvent 
appara\^\i tre au d\'ebut d'une paire de lignes de message d'erreur~:
\ulist
\li ^|<output>| indique que \TeX\ \'etait au milieu de sa routine de sortie 
quand cette erreur s'est produite.
\li ^|<write>| indique que \TeX\ \'etait en train d'ex\'ecuter une commande 
|\write| quand cette erreur s'est produite. \TeX\ d\'etectera une telle 
erreur quand il fera r\'eellement le |\write| (pendant un |\shipout|), et 
non quand il rencontre le |\write|.
\endulist


\eix^^{messages d'erreur}
\endchapter
\byebye

