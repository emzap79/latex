% vim:ts=2:expandtab
% GIT Quick Reference Card
% Copyright (c) 2011 Jonas Petong.
% TeX Format - Based on VIM Quick Reference Card

% compile as 'pdftex git-qrc.tex'
% preamble%{{{
\def\versionnumber{1.0}  % Version of this reference card
\def\year{2014}
\def\month{Sep}
\def\version{v\versionnumber\ \month\ \year}

\def\shortcopyrightnotice{\smallskip
  \centerline{\smallrm \copyright\ \year\ Jonas Petong -
  Permissions on back. \version}}

\def\copyrightnotice{
\vfill \hrule\smallskip\begingroup\smallrm
\centerline{\version. Copyright \copyright\ \year\ Jonas Petong}

Permission is granted to make and distribute copies of this card provided
the copyright notice and this permission notice are preserved on all
copies. Send comments or corrections to Jonas Petong at:
$\langle$kaktuuus@gmx.de$\rangle$

\centerline{http://ebixio.com/ and http://github.com/gburca/git-qrc}
\endgroup}

% \pdfoutput=1
\pdfpageheight=21cm
\pdfpagewidth=29.7cm

% Font definitions
\font\bigbf=cmbx12
\font\smallrm=cmr8
\font\smalltt=cmtt8
\font\tinyit=cmmi5

\def\title#1{\hfil{\bf #1}\hfil\par\vskip 2pt\hrule}
\def\section#1{\vskip 0.7cm {\it{\bf#1}\/}\par\vskip 1pt}
\def\subsection#1{\vskip 0.7cm {\it#1\/}\par\vskip 1pt\hrule\vskip 1pt}


% Characters definitions
\def\\{\hfil\break}
\def\bs{$\backslash$}
\def\backspace{$\leftarrow$}
\def\ctrl{{\rm\char94}\kern-1pt}
\def\enter{$\hookleftarrow$}
\def\or{\thinspace{\tinyit{or}}\thinspace}
\def\key#1{$\langle${\rm{\it#1\/}}$\rangle$}
\def\rapos{\char125}
\def\lapos{\char123}
\def\bs{\char92}
\def\tild{\char126}
\def\lbracket{[}
\def\rbracket{]}

% Three columns definitions
\parindent 0pt
\nopagenumbers
\hoffset=-1.56cm
\voffset=-1.54cm
\newdimen\fullhsize
\fullhsize=27.9cm
\hsize=8.5cm
\vsize=19cm
\def\fullline{\hbox to\fullhsize}
\let\lr=L
\newbox\leftcolumn
\newbox\midcolumn
\output={
  \if L\lr
    \global\setbox\leftcolumn=\columnbox
    \global\let\lr=M
  \else\if M\lr
    \global\setbox\midcolumn=\columnbox
    \global\let\lr=R
  \else
    \tripleformat
    \global\let\lr=L
  \fi\fi
  \ifnum\outputpenalty>-20000
  \else
    \dosupereject
  \fi}
\def\tripleformat{
  \shipout\vbox{\fullline{\box\leftcolumn\hfil\box\midcolumn\hfil\columnbox}}
  \advancepageno}
\def\columnbox{\leftline{\pagebody}}

% GIT optimizations
\def\cmdFooBar#1{{\tt#1}\null}	%\null so not an abbrev even if period follows

% A short command
\def\cmd#1#2{
  \noindent
  \hbox to \hsize {% This comment is needed to remove space at start of line
    \vtop{
      \hbox to 2.4cm {
        \noindent\cmdFooBar{#1}\dotfill
      }
      %\hrule
    }% This comment is needed to remove space
    \hfil	% absorb extra space b/w columns
    \vtop{
      \hsize=5.90cm
      \hbox{\hfuzz = 15pt \vtop{
      % Uncomment line below to right-align description
      %\leftskip=0cm plus 150fil
      {#2}
      }}
      %\hrule
    }
  }
  %\hrule
  \par
  \vskip 0.14cm
}

\def\cmdExample#1#2#3{
      \hbox to 1.4cm {
      % \hfuzz = 5pt
      \hskip 14pt
      %\hfil
      %\hsize=5.9cm
      \noindent\cmdFooBar{\it#1}
      }
      \hbox to 2.4cm {
      % \hfuzz = 5pt
      \hskip 14pt
      %\hfil
      %\hsize=5.9cm
      \noindent\cmdFooBar{\it#2}
      }
}

% A short command + options
\def\cmdOpt#1#2{
  \noindent
  \hbox to \hsize {% This comment is needed to remove space at start of line
    \vtop{
      \hbox to 1.7cm {
      % \hfuzz = 5pt
      \hskip 14pt
      %\hfil
      %\hsize=5.9cm
      \noindent\cmdFooBar{\it#1}
      }
      %\hrule
    }% This comment is needed to remove space
    \hfil	% absorb extra space b/w columns
    \vtop{
      \hsize=5.90cm
      \hbox{\hfuzz = 15pt \vtop{
      % Uncomment line below to right-align description
      %\leftskip=0cm plus 150fil
      \it{#2}
      }}
      %\hrule
    }
  }
  %\hrule
  \par
  \vskip 0.14cm
}

% A long command
\def\cmdL#1#2{
  \hsize=8.5cm
  \vbox {
    \hbox{
        \cmdFooBar{#1}\hfil
    }%
    %\vskip -0.2cm % adjusts space b/w command & description
    \hskip 2.5cm  % indent for description
    \hbox to 5.9cm {%
      \hfuzz = 5pt
      % Use the first one when right-aligned
      %\vrule\hskip -0.05cm
      %\vrule\hskip 0.05cm
      \hfil
      \hsize=5.9cm
      \vtop{
        %\hrule
        % Uncomment line below to right-align description
        %\leftskip=0cm plus 150fil
        {#2}
        }}
  }%
  %\hrule
  \par
  \vskip 0.14cm
}
%}}}
% card content%{{{
% Header
\title{BASH QUICK REFERENCE CARD}

\shortcopyrightnotice

\section{History}
\subsection{Shortcuts}
\cmd{\ctrl p}{Fetch the previous command from the history list}
\cmd{\ctrl n}{Fetch the next command from the history list}
\cmd{\ctrl r}{Search history backward (incremental search)}
\cmd{\ctrl s}{Search history forward (incremental search)}
\cmd{\ctrl r}{Search history backward for a command matching \it{string}}
\cmd{\ctrl s}{Search history forward for a command matching \it{string}}
\cmd{Meta-p}{Search backward using non-incremental search}
\cmd{Meta-n}{Search forward using non-incremental search}
\cmd{Meta-$<$}{Move to the first line in the history}
\cmd{Meta-$>$}{Move to the end of the history list}
\cmd{k}{Fetch the previous command from the history list}
\cmd{j}{Fetch the next command from the history list}
\cmd{n}{Repeat search in the same direction as previous}
\cmd{N}{Repeat search in the opposite direction as previous}
\cmd{G}{Move to the N-th history line}
\cmdOpt{15G}{move to $15$th line}

\subsection{History-Commands}
example commmand is: \tt{cp foo bar}\\
\cmd{!$n$}{executes command number $n$}
\cmd{!!}{runs previous executed command (also known as `bang bang')}
% \cmdExample{echo foo}{previous history entry}
\cmdOpt{echo foo}{previous history entry}
\cmdOpt{sudo\ !!}{sudo echo foo}
\cmd{!$c$}{executes previous command, which began on $c$}
\cmdOpt{!ls}{ls -l ~/foobar}
\cmd{$ex$!*}{previous command without the first word}
\cmdOpt{mv !*}{mv foo}
\cmd{vim !{{\rm\char94}\kern-1pt}}{vim foo}
\cmd{!n}{Refers to the {\bf n}-th command line}
\cmd{!-n}{Refers to the current command line minus {\bf n}}
\cmd{!string}{Refers to the most recent command starting with {\bf string}}
\cmd{!?string?}{Refers to the most recent command containing {\bf string} (the ending ? is optional)}
\cmdL{\^{}string1\^{}string2\^{}}{Quick substitution. Repeats the last command, replacing {\bf string1} with {\bf string2}}
\cmd{!\#}{Refers to the entire command line typed so far}
\cmd{0}{The zeroth (first) word in a line (usually command name)}
\cmd{n}{The {\bf n}-th word in a line}
\cmd{\^{}}{The first argument (the second word) in a line}
\cmd{\$}{The last argument in a line}
\cmd{\%}{The word matched by the most recent ?string? search}
\cmd{x-y}{A range of words from {\bf x} to {\bf y} ({\bf -y} is synonymous with {\bf 0-y})}
\cmd{*}{All word but the zeroth}
\cmd{x*}{Synonymous with {\bf x-\$}}
\cmd{x-}{The words from {\bf x} to the second to last word}
\cmd{h}{Removes a trailing pathname component, leaving the head}
\cmd{t}{Removes all leading pathname components, leaving the tail}
\cmd{r}{Removes a trailing suffix of the form .xxx, leaving the basename}
\cmd{e}{Removes all but the trailing suffix}
\cmd{p}{Prints the resulting command but does not execute it}
\cmd{q}{Quotes the substituted words, escaping further substitutions}
\cmd{x}{Quotes the substituted words, breaking them into words at blanks and newlines}
\cmd{s/old/new/}{Substitutes {\bf new} for {\bf old}}
\cmd{\&}{Repeats the previous substitution}
\cmd{g}{Causes {\bf s/old/new/} or {\bf \&} to be applied over the entire event line}

% \large{\bf Shell Variable} & \large{\bf Description}
\subsection{Variables}
\cmd{HISTFILE}{Controls where the history file gets saved. Set to {\bf/dev/null} not to keep history (Default: $\sim$/.bash\_history)}
\cmd{HISTFILESIZE}{Controls how many history commands to keep in {\bf HISTFILE} (Default: 500)}
\cmd{HISTSIZE}{Controls how many history commands to keep in the history list of current session (Default: 500)}
\cmd{HISTIGNORE}{Controls which commands to ignore and not save to the history list. The variable takes a list of colon separated patterns. Pattern {\bf \&} matches the previous history command}

% \large{\bf shopt option} & \large{\bf Description}
\subsection{Shopt Option}
shopt options can be set by a \tt{shopt -s option} and can be unset by a \tt{shopt -u option} shell command.\\
\cmd{histappend}{Setting the variable appends current session history to {\bf HISTFILE}. Unsetting overwrites the file each time}
\cmd{histreedit}{If set, puts a failed history substitution back on the command line for re-editing}
\cmd{histverify}{If set, puts the command to be executed after a substitution on command line as if you had typed it}
% A cheat sheet by {\bf Peteris Krumins} (peter@catonmat.net), 2008.

% \large{\bf Shortcut} & \large{\bf Description}


%% Footer
\copyrightnotice

% Ending
\supereject
\if L\lr \else\null\vfill\eject\fi
\if L\lr \else\null\vfill\eject\fi
\bye

% EOF%}}}
