#LyX 2.0 created this file. For more info see http://www.lyx.org/
\lyxformat 413
\begin_document
\begin_header
\textclass article
\begin_preamble
\@ifundefined{definecolor}{\usepackage{color}}{}
% This file was converted to LaTeX by Writer2LaTeX ver. 1.0.2
% see http://writer2latex.sourceforge.net for more info

\usepackage[noenc]{tipa}\usepackage{tipx}\usepackage[geometry,weather,misc,clock]{ifsym}\usepackage{pifont}\usepackage{eurosym}\usepackage{wasysym}\usepackage{amsfonts}\@ifundefined{definecolor}{\usepackage{color}}{}
\usepackage{array}\usepackage{hhline}% Page layout (geometry)
\setlength{\voffset}{-1in}
\setlength{\hoffset}{-1in}
\setlength{\topmargin}{2cm}
\setlength{\oddsidemargin}{2cm}
\setlength{\textheight}{24.599998cm}
\setlength{\textwidth}{16.999cm}
\setlength{\footskip}{1.099cm}
\setlength{\headheight}{0cm}
\setlength{\headsep}{0cm}
% Footnote rule
\setlength{\skip\footins}{0.119cm}
\renewcommand{\footnoterule}{\vspace*{-0.018cm}\setlength\leftskip{0pt}\setlength\rightskip{0pt plus 1fil}\noindent\textcolor{black}{\rule{0.25\columnwidth}{0.018cm}}\vspace*{0.101cm}}
% Pages styles

\newcommand{\ps@Standard}{
  \renewcommand\@oddhead{}
  \renewcommand\@evenhead{}
  \renewcommand\@oddfoot{\sffamily \thepage{}}
  \renewcommand\@evenfoot{\@oddfoot}
  \renewcommand\thepage{\arabic{page}}
}
\newcommand{\ps@Index}{
  \renewcommand\@oddhead{}
  \renewcommand\@evenhead{}
  \renewcommand\@oddfoot{\textstyleInhaltsverzeichnisFuzeile{\thepage{}}}
  \renewcommand\@evenfoot{\@oddfoot}
  \renewcommand\thepage{\arabic{page}}
}
\newcommand{\ps@FirstPage}{
  \renewcommand\@oddhead{}
  \renewcommand\@evenhead{}
  \renewcommand\@oddfoot{}
  \renewcommand\@evenfoot{}
  \renewcommand\thepage{\arabic{page}}
}


\title{write.hausarbeit}
\author{Jonas Petong}
\date{17.10.2011}
\end_preamble
\use_default_options false
\begin_modules
customHeadersFooters
\end_modules
\maintain_unincluded_children false
\language ngerman
\language_package default
\inputencoding utf8
\fontencoding global
\font_roman default
\font_sans default
\font_typewriter default
\font_default_family default
\use_non_tex_fonts false
\font_sc false
\font_osf false
\font_sf_scale 100
\font_tt_scale 100

\graphics default
\default_output_format default
\output_sync 0
\bibtex_command default
\index_command default
\paperfontsize default
\spacing single
\use_hyperref true
\pdf_title "Dokument"
\pdf_author "Jonas Petong"
\pdf_bookmarks true
\pdf_bookmarksnumbered false
\pdf_bookmarksopen false
\pdf_bookmarksopenlevel 1
\pdf_breaklinks false
\pdf_pdfborder false
\pdf_colorlinks true
\pdf_backref section
\pdf_pdfusetitle true
\pdf_quoted_options "pdftex,linkcolor=blue,citecolor=blue,filecolor=blue,urlcolor=blue,pdfsubject=,pdfkeywords="
\papersize a4paper
\use_geometry false
\use_amsmath 2
\use_esint 1
\use_mhchem 1
\use_mathdots 1
\cite_engine basic
\use_bibtopic false
\use_indices false
\paperorientation portrait
\suppress_date false
\use_refstyle 0
\index Stichwortverzeichnis
\shortcut idx
\color #008000
\end_index
\secnumdepth 3
\tocdepth 3
\paragraph_separation indent
\paragraph_indentation default
\quotes_language danish
\papercolumns 1
\papersides 1
\paperpagestyle default
\tracking_changes false
\output_changes false
\html_math_output 0
\html_css_as_file 0
\html_be_strict false
\end_header

\begin_body

\begin_layout Standard
\begin_inset Newpage clearpage
\end_inset


\begin_inset ERT
status open

\begin_layout Plain Layout


\backslash
thispagestyle{FirstPage}
\end_layout

\end_inset


\end_layout

\begin_layout Standard
\begin_inset VSpace bigskip
\end_inset


\end_layout

\begin_layout Title

\series bold
-- Praktikumsbericht -- 
\end_layout

\begin_layout Standard
\begin_inset VSpace bigskip
\end_inset

 
\begin_inset VSpace bigskip
\end_inset

 
\begin_inset VSpace bigskip
\end_inset

 
\begin_inset VSpace bigskip
\end_inset

 
\begin_inset VSpace bigskip
\end_inset

 
\begin_inset VSpace bigskip
\end_inset


\end_layout

\begin_layout Standard
\begin_inset ERT
status open

\begin_layout Plain Layout

{
\end_layout

\end_inset


\begin_inset ERT
status open

\begin_layout Plain Layout


\backslash
centering
\end_layout

\end_inset


\series bold
\size large
Praktikum bei der
\end_layout

\begin_layout Standard

\series bold
\begin_inset VSpace bigskip
\end_inset


\end_layout

\begin_layout Standard

\series bold
\begin_inset ERT
status open

\begin_layout Plain Layout

{
\end_layout

\end_inset


\begin_inset ERT
status open

\begin_layout Plain Layout


\backslash
centering
\end_layout

\end_inset


\series default
\size large

\begin_inset Quotes ard
\end_inset

Cooperativa de la Policía Nacional
\series bold
\size default

\begin_inset Quotes ard
\end_inset


\begin_inset ERT
status open

\begin_layout Plain Layout

}
\end_layout

\end_inset


\end_layout

\begin_layout Standard

\series bold
\begin_inset VSpace bigskip
\end_inset


\end_layout

\begin_layout Standard

\series bold
\begin_inset ERT
status open

\begin_layout Plain Layout

{
\end_layout

\end_inset


\begin_inset ERT
status open

\begin_layout Plain Layout


\backslash
centering
\end_layout

\end_inset


\size large
in Quito, Ecuador
\series default
 
\size default

\begin_inset ERT
status open

\begin_layout Plain Layout

}
\end_layout

\end_inset


\end_layout

\begin_layout Standard
\begin_inset VSpace bigskip
\end_inset

 
\begin_inset VSpace bigskip
\end_inset

 
\begin_inset VSpace bigskip
\end_inset

 
\begin_inset VSpace bigskip
\end_inset

 
\begin_inset VSpace bigskip
\end_inset

 
\begin_inset VSpace bigskip
\end_inset

 
\begin_inset VSpace bigskip
\end_inset

 
\begin_inset VSpace bigskip
\end_inset

 
\begin_inset VSpace bigskip
\end_inset

 
\begin_inset VSpace bigskip
\end_inset

 
\begin_inset VSpace bigskip
\end_inset

 
\begin_inset VSpace bigskip
\end_inset

 
\begin_inset VSpace bigskip
\end_inset

 
\begin_inset VSpace bigskip
\end_inset

 
\begin_inset VSpace bigskip
\end_inset

 
\begin_inset VSpace bigskip
\end_inset

 
\begin_inset VSpace bigskip
\end_inset

 
\begin_inset VSpace bigskip
\end_inset

 
\begin_inset VSpace bigskip
\end_inset

 
\begin_inset VSpace bigskip
\end_inset

 
\begin_inset VSpace bigskip
\end_inset

 
\begin_inset VSpace bigskip
\end_inset

 
\begin_inset VSpace bigskip
\end_inset

 
\begin_inset VSpace bigskip
\end_inset

 
\begin_inset VSpace bigskip
\end_inset

 
\begin_inset VSpace bigskip
\end_inset

 
\begin_inset VSpace bigskip
\end_inset

 
\begin_inset VSpace bigskip
\end_inset

 
\begin_inset VSpace bigskip
\end_inset

 
\begin_inset VSpace bigskip
\end_inset

 
\begin_inset VSpace bigskip
\end_inset

 
\begin_inset VSpace bigskip
\end_inset

 
\begin_inset VSpace bigskip
\end_inset


\end_layout

\begin_layout Standard
\begin_inset ERT
status open

\begin_layout Plain Layout

}
\end_layout

\end_inset


\end_layout

\begin_layout Standard
\begin_inset ERT
status open

\begin_layout Plain Layout


\backslash
hspace{0.55
\backslash
paperwidth}
\end_layout

\end_inset


\begin_inset ERT
status open

\begin_layout Plain Layout

{
\end_layout

\end_inset


\begin_inset ERT
status open

\begin_layout Plain Layout


\backslash
color{black}
\end_layout

\end_inset

 Jonas Petong
\begin_inset ERT
status open

\begin_layout Plain Layout

}
\end_layout

\end_inset


\end_layout

\begin_layout Standard
\align block
\begin_inset ERT
status open

\begin_layout Plain Layout


\backslash
hspace{0.55
\backslash
paperwidth}
\end_layout

\end_inset


\begin_inset ERT
status open

\begin_layout Plain Layout

{
\end_layout

\end_inset


\begin_inset ERT
status open

\begin_layout Plain Layout


\backslash
color{black}
\end_layout

\end_inset

 Matr.Nr.: 2224159
\begin_inset ERT
status open

\begin_layout Plain Layout

}
\end_layout

\end_inset


\end_layout

\begin_layout Standard
\begin_inset ERT
status open

\begin_layout Plain Layout


\backslash
hspace{0.55
\backslash
paperwidth}
\end_layout

\end_inset


\begin_inset ERT
status open

\begin_layout Plain Layout

{
\end_layout

\end_inset


\begin_inset ERT
status open

\begin_layout Plain Layout


\backslash
color{black}
\end_layout

\end_inset

 44789 Bochum
\begin_inset ERT
status open

\begin_layout Plain Layout

}
\end_layout

\end_inset


\end_layout

\begin_layout Standard
\begin_inset ERT
status open

\begin_layout Plain Layout


\backslash
hspace{0.55
\backslash
paperwidth}
\end_layout

\end_inset


\begin_inset ERT
status open

\begin_layout Plain Layout

{
\end_layout

\end_inset


\begin_inset ERT
status open

\begin_layout Plain Layout


\backslash
color{black}
\end_layout

\end_inset

 Tel.: 0176 – 533 47 208
\begin_inset ERT
status open

\begin_layout Plain Layout

}
\end_layout

\end_inset


\end_layout

\begin_layout Standard
\begin_inset ERT
status open

\begin_layout Plain Layout


\backslash
hspace{0.55
\backslash
paperwidth}
\end_layout

\end_inset


\begin_inset ERT
status open

\begin_layout Plain Layout

{
\end_layout

\end_inset


\begin_inset ERT
status open

\begin_layout Plain Layout


\backslash
color{black}
\end_layout

\end_inset

 E-Mail: 
\begin_inset ERT
status open

\begin_layout Plain Layout


\backslash
href{mailto:jonas.petong@stud.uni-due.de}{
\end_layout

\end_inset

jonas.petong@stud.uni-due.de
\begin_inset ERT
status open

\begin_layout Plain Layout

}
\end_layout

\end_inset


\begin_inset ERT
status open

\begin_layout Plain Layout

}
\end_layout

\end_inset


\end_layout

\begin_layout Standard
\begin_inset ERT
status open

\begin_layout Plain Layout


\backslash
hspace{0.55
\backslash
paperwidth}
\end_layout

\end_inset


\begin_inset ERT
status open

\begin_layout Plain Layout

{
\end_layout

\end_inset


\begin_inset ERT
status open

\begin_layout Plain Layout


\backslash
color{black}
\end_layout

\end_inset

 BA-Volkswirt, 5.
 Fachsemester
\begin_inset ERT
status open

\begin_layout Plain Layout

}
\end_layout

\end_inset


\end_layout

\begin_layout Standard
\begin_inset Newpage clearpage
\end_inset


\end_layout

\begin_layout Standard
\begin_inset CommandInset toc
LatexCommand tableofcontents

\end_inset


\end_layout

\begin_layout Standard
\begin_inset Newpage newpage
\end_inset


\end_layout

\begin_layout Section
Unternehmensporträt
\end_layout

\begin_layout Standard
\align block
Die Kooperation 
\begin_inset Quotes eld
\end_inset

Cooperativa de Ahorro y Crédito Policía Nacional Ltda.
 (CPN)
\begin_inset Quotes erd
\end_inset

 wurde am 26.
 Juni 1976 von 35 Polizeimitgliedern auf der 
\begin_inset Quotes eld
\end_inset

Plaza de Quito
\begin_inset Quotes erd
\end_inset

 ins Leben gerufen.
 Grundgedanke des Zusammenschlusses war es, Mitarbeitern der Polizei Kredite
 und Darlehen zu günstigen Konditionen zu ermöglichen, um so ihre Kollegen
 zu unterstützen.
 Heute befindet sich das Hauptgebäude im 
\begin_inset Quotes ald
\end_inset

Edificio Matriz
\begin_inset Quotes ard
\end_inset

 auf der Avenida 10 de Agosto.
 Das Gebäude erstreckt sich über sieben Stockwerke und bietet Raum für 239
 Mitarbeiter.
 Seit dem Jahr 2002 gibt es im besonderen einige Faktoren, die zu einem
 beständigen Wachstum der CPN beitrugen:
\end_layout

\begin_layout Enumerate

\series bold
Personal: 
\series default
Seit ihrem Beginn im Jahr 1976 hat sich der Personalstand mehr als verzwanzigfac
ht.
 Heute werden Mitarbeiter aus allen Bereichen des betriebs- sowie volkswirtschaf
tlichen Bereichen eingesetzt um die CPN in ihrer Arbeit zu unterstützen.
\end_layout

\begin_layout Enumerate

\series bold
Technische Ausrüstung und Infrastruktur: 
\series default
Unter dem Stichwort 
\begin_inset Quotes ald
\end_inset

Technik
\begin_inset Quotes ard
\end_inset

 wird die Verbesserung der Kommunikation und Interaktion zwischen den Mitglieder
n und Mitarbeitern der Cooperativa verstanden.
 Der wichtigste aktueller Bestandteil neuer technischer Weiterentwicklung
 ist dabei die Einrichtung von Geldautomaten und die Implementierung des
 Internet Banking.
 Neben dem Online-Banking soll dies, in einer späteren Entwicklungsstufe
 der Software, dazu dienen, Wünsche und Anregungen der Mitglieder bezüglich
 der angebotenen Leistungen und Produkte, zeitnah zu beantworten.
 Hierbei wird besonderer Schwerpunkt gelegt auf die eindeutige Identifizierung
 der Nutzer mittels Fingerabdruck, um etwaige Fehler zu umgehen und die
 Sicherheit der Geldtransfers zu optimieren.
\end_layout

\begin_layout Enumerate

\series bold
Service: 
\series default
Neben der Ausarbeitung des Online-Bankings ist die CPN bestrebt etwaige
 Unklarheiten in der Auseinandersetzung mit dem Mitglied zu besprechen.
 Dazu gehören zum einen telefonische Ansprechstellen, zum anderen die Möglichkei
t des Vier-Augen-Gesprächs welches nach Absprache realisiert werden kann.
 Bei einem persönlichen Termin können die Anliegen des Mitglieds besprochen,
 und auf Fragen bezüglich der von der CPN angebotenen Leistungen eingegangen
 werden.
\end_layout

\begin_layout Standard
Benannte Schwerpunkte haben zu einem stetigen Wachstum der CPN in den letzten
 9 Jahren geführt, der sich in der in der Einrichtung zahlreicher Niederlassunge
n im ganzen Land widerspiegelt.
 Darüber hinaus wurde ein Produktkatalog mit den Leistungen erstellt, der
 den Mitgliedern eine Übersicht über die angebotenen Leistungen bietet.
 Mit mehr als 48.000 Mitgliedern ist die 
\begin_inset Quotes ald
\end_inset

Cooperativa de la Policía Nacional
\begin_inset Quotes ard
\end_inset

 die größte Bank ihrer Art in Ecuador.
\end_layout

\begin_layout Section
Formalitäten
\end_layout

\begin_layout Subsection
Beantragung eines Kontos
\end_layout

\begin_layout Standard
Für die Einrichtung eines neuen Kontos muss der Interessent sich in eine
 der Geschäftsstellen der CPN begeben, zusammen mit den notwendigen Unterlagen.
 Mitgebracht werden sollten eine Kopie des Personalausweises, eines Beglaubigung
sschreibens, Einer Fotografie im Personalausweisformat und 25 Dollar Anfangsguth
aben.
 Diese können auch mit der ersten monatlichen Überweisung eingezahlt werden.
\end_layout

\begin_layout Subsection
Kreditantragstellung
\end_layout

\begin_layout Standard
Um einen Kredit von der CPN zu erhalten muss der Antragsteller auf der Webseite
 der 
\shape slanted
Cooperativa de la Policía Nacional
\shape default
 ein elektr.
 Formblatt ausfüllen.
 Seine Anfrage wird auf elektr.
 Wege bearbeitet und der Antragsteller wird per EMail über das Datum sowie
 den Ort der Einzahlungen informiert.
 Jeden Monat behält die CPN einen Teil der Tilgung des Kredites für sich
 ein.
 Diese Summe beträgt 12 $ und gliedert sich wie folgt in drei Teile:
\end_layout

\begin_layout Itemize
7,40$ werden für das Sparbuch verwendet, das dem Kreditnehmer als Grundlage
 für seinen Kredit dient.
 2$ gehen auf ein weiteres Konto, von welchem der Kreditnehmer jederzeit
 Geld abheben kann.
 2,60$ werden für eine Gesundheits- und Todesversicherung verwendet.
 Der Mindestbeitrag für den Fond beträgt 300$.
 Er garantiert allen Mitgliedern ein Anrecht auf Hilfeleistungen im Krankheits-
 oder Todesfall.
\end_layout

\begin_layout Subsubsection
Konditionen
\end_layout

\begin_layout Standard
Die Unterhaltung eines Kontos ist für alle Mitglieder der Polizei gratis.
 Die Zinsen der von der CPN angebotenen Kredite sind um ein vielfaches niedriger
 als bei anderen Banken oder Kreditinstituten.
 Für den 
\begin_inset Quotes ald
\end_inset

Crédito Ordinario
\begin_inset Quotes ard
\end_inset

 und den 
\begin_inset Quotes ald
\end_inset

Crédito Emergente
\begin_inset Quotes ard
\end_inset

 betragen die Zinsen 12,94%.
 Ein vergleichbarer Kredit bei einer privaten Bank in Ecuador ist nur zu
 einem Zinssatz von 15,20% zu erhalten.
\begin_inset Foot
status collapsed

\begin_layout Plain Layout
BGR NET (2011), 
\shape italic
BGR Créditos
\shape default
, Webseite: 
\begin_inset ERT
status open

\begin_layout Plain Layout


\backslash
href {
\end_layout

\end_inset

http://www.bgr.com.ec/portal/web/guest/bgr-consumo1
\begin_inset ERT
status open

\begin_layout Plain Layout

}{www.bgr.com.ec}
\end_layout

\end_inset

, zugegriffen am 23.
 Oktober 2011, 15:09 MEZ.
\end_layout

\end_inset

 
\end_layout

\begin_layout Standard
Die Zinsen, die auf ein Sparkonto der CPN gewährt werden liegen zwischen
 7 und 10 %.
 Ausschlaggebend für den Prozentsatz ist die Höhe und festgelegte Laufzeit
 des Kontos.
\end_layout

\begin_layout Subsubsection
Bürgschaft
\end_layout

\begin_layout Standard
Zur Beantragung eines Kredits muss der Debitor einen Bürgen aufbringen,
 der für die Rückzahlung des aufgenommenen Kredits bürgt.
 Die Voraussetzungen, damit eine Person als Bürge eingesetzt werden darf
 ist auch davon abhängig, wieviele Jahre der Kreditnehmer für die Polizei
 gedient hat:
\end_layout

\begin_layout Itemize
Hat der Antragsteller weniger als 10 Jahre Dienst geleistet, so braucht
 er einen Bürgen, der mindestens 10 Jahre bei der Polizei gedient hat oder
 nachweislich pensioniert ist.
 Findet er keine Person, die eine der beiden Bedingungen erfüllt, so können
 auch zwei Kollegen als Bürgen auftreten, die beide mehr als drei Jahre
 bei der 
\shape italic
Policía Nacional 
\shape default
gedient haben.
\end_layout

\begin_layout Itemize
Hat der Antragsteller mehr als zehn Jahre für die Polizei gedient, braucht
 er einen Bürgen, der mehr als fünf Jahre bei der Polizei gedient hat oder
 nachweislich pensioniert ist.
\end_layout

\begin_layout Itemize
Hat er mehr als 25 Jahre für die Polizei gearbeitet, so bedarf es gar keinen
 Bürgen.
\end_layout

\begin_layout Standard
\begin_inset Newpage newpage
\end_inset


\end_layout

\begin_layout Section

\series bold
Auflistung angebotener Leistungen
\end_layout

\begin_layout Standard
Zwischen den von der CPN angebotenen Krediten gibt es Unterschiede, sowohl
 in ihrer Höhe, als auch in den Voraussetzungen, die zur Beantragung des
 jeweiligen Kredits erfüllt sein müssen.
 Im Folgenden sollen die Wichtigsten aufgeführt werden.
\end_layout

\begin_layout Subsection
Hypothekendarlehen (
\begin_inset Quotes ald
\end_inset

Crédito Hipotecario
\begin_inset Quotes ard
\end_inset

)
\end_layout

\begin_layout Standard
\begin_inset Wrap figure
lines 0
placement o
overhang 0col%
width "50col%"
status open

\begin_layout Plain Layout

\end_layout

\begin_layout Plain Layout
\begin_inset Caption

\begin_layout Plain Layout
Hypothekenkredit
\end_layout

\end_inset


\end_layout

\begin_layout Plain Layout
\begin_inset Graphics
	filename hipotecario.jpg

\end_inset


\end_layout

\end_inset


\end_layout

\begin_layout Standard

\end_layout

\begin_layout Standard
Zur 
\series bold
Beantragung eines Hypothekendarlehens
\series default
 ist eine Kopie des Personalausweises (
\begin_inset Quotes ald
\end_inset

Copia de la cédula de ciudadana
\begin_inset Quotes ard
\end_inset

) sowie eine Bescheinigung über die Wahlberechtigung erforderlich.
 Darüberhinaus muss ein 
\begin_inset Quotes ald
\end_inset

Orden de Avalúo
\begin_inset Quotes ard
\end_inset

 gestellt werden, vergleichbar mit der deutschen SCHUFA-Auskunft.
 Außerdem muss der Antragsteller Bescheinigungen über die letzten Bankauszüge
 sowie seine Verdienste und eine Liste der Strom-, Wasser- und Gasabrechnungen
 aufzeigen.

\series bold
 
\series default
Von Seiten des Ehepartners
\series bold
 
\series default
des Antragstellers oder der Antragstellerin muss gleichermaßen eine Bescheinigun
g über die Wahlberechtigung vorliegen.
 Auch müssen Zeugnisse über die letzten Kontoaktivitäten der Ehegattin oder
 des Ehegatten vorliegen.
 
\end_layout

\begin_layout Standard
Kreditanträge können gestellt werden für folgende Angelegenheiten:
\end_layout

\begin_layout Itemize
Kauf einer neuen Immobilie
\end_layout

\begin_layout Itemize
Kauf eines Grundstücks oder einer gebrauchten Immobilie.
\end_layout

\begin_layout Itemize
Neubau eines Hauses
\end_layout

\begin_layout Itemize
Erweiterung des Eigenheims
\end_layout

\begin_layout Itemize
Bauabschluss
\end_layout

\begin_layout Standard
Im Falle des Neubaus, eines Erweiterungsbaus oder bei Bauabschluss des Eigenheim
s muss der Kreditnehmer innerhalb von drei Monaten eine Bescheinigung über
 die Nutzung der entsprechenden Immobilie abgeben.
 Wird dem Anliegen des Antragstellers stattgegeben, so erhält er einen Hypotheke
nkredit mit einer Laufzeit von 15 Jahren.
\end_layout

\begin_layout Subsection

\series bold
Allgemeiner Kredit (
\begin_inset Quotes ald
\end_inset

Credito Ordinario 
\begin_inset Quotes ald
\end_inset

)
\end_layout

\begin_layout Standard
Der 
\begin_inset Quotes ald
\end_inset

Credito Ordinario
\begin_inset Quotes ard
\end_inset

 ist befristet auf 48 Monate und kann von allen Mitgliedern der CPN beantragt
 werden.
 Der höchste zu beantragende Betrag beträgt 8.000 Dollar.
 Der Credito Ordinario wird frühestens 48 Stunden nach Einreichen aller,
 im Folgenden benannten, Bedingungen geprüft und akkreditiert.
\end_layout

\begin_layout Standard
Zu den 
\series bold
Bedingungen 
\series default
des Allgemeinen Kredites gehören:
\end_layout

\begin_layout Itemize
Regelmäßiger Besuch der Internetseite 
\begin_inset ERT
status collapsed

\begin_layout Plain Layout


\backslash
href{http://www.cooperando.fin.ec}{www.cooperando.fin.ec}
\end_layout

\end_inset

 zu Anfang jeden Monats.
\end_layout

\begin_layout Itemize
Kopie des Personalausweises und einer Beglaubigungsbescheinigung, sowohl
 des Kreditnehmers als auch des Bürgen.
\end_layout

\begin_layout Itemize
Bescheinigung über die Wahlberechtigung.
\begin_inset Foot
status open

\begin_layout Plain Layout
In Ecuador muss jede volljährige Person über 18 Jahre eine Wahlberechtigung
 beantragen.
 Die Teilnahme an Nationalen Parlamentswahlen ist in Ecuador gesetzlich
 vorgeschrieben und Pflicht.
\end_layout

\end_inset


\end_layout

\begin_layout Itemize
Bürgschaft eines rechtmäßig befähigten Bürgen.
 Rechtmäßige Bürgen sind alle volljährigen Personen, welche ein Einkommen
 von wenigstens 400 $ im Monat nachweisen können:
\end_layout

\begin_deeper
\begin_layout Itemize
Dienstältere Mitglieder, welche 18 Jahre und mehr bei der 
\begin_inset Quotes ald
\end_inset

Policía Nacional
\begin_inset Quotes ard
\end_inset

 gearbeitet haben, müssen keinen Bürgen vorbringen, um einen Kredit zu beantrage
n.
\end_layout

\begin_layout Itemize
Polizisten im inneren Dienst, deren Einkommen mehr als $ 400 Dollar beträgt
 brauchen ebenfalls keinen Bürgen zur Beantragung des Kredits.
\end_layout

\end_deeper
\begin_layout Standard
Die 
\series bold
Vorteile für Kreditnehmer
\series default
 eines 
\begin_inset Quotes ald
\end_inset

Crédito Ordinarios
\begin_inset Quotes ard
\end_inset

 sind, unter anderem günstigere Kreditkonditionen im Vergleich zum herkömmlichen
 Bankkredit; Gutschrift des Darlehens auf das Sparkonto der 
\begin_inset Quotes ald
\end_inset

Cooperativa
\begin_inset Quotes ard
\end_inset

; Versicherung gegen Wertverlust; Möglichkeit der Online-Einzahlung der
 fälligen Raten über das Internet; Zeitersparnis wegen Wegfall unnötigen
 bürokratischen Aufwands und überflüssiger Bankengänge; Erneuerung des Kredits
 nach 80% des beglichenen alten Kredits sowie Zahlungsfristen bis zu 36
 Monaten.
\end_layout

\begin_layout Subsection
Mikrokredit (
\begin_inset Quotes ald
\end_inset

Microcrédito
\begin_inset Quotes ard
\end_inset

)
\end_layout

\begin_layout Standard
Mikrokredite sind Darlehen oder Kredite, die an Menschen vergeben werden,
 welche, durch Ihre ökonomische Situation und ihr niedriges Einkommensniveau
 keinen Kredit bei einer traditionellen Bank erhalten würden.
 Mikrokredite ermöglichen vor allem in Entwicklungsländern, dass Personen
 ohne die benötigte finanzielle Grundausstattung in Arbeitsprojekte auf
 ihre eigene Rechnung investieren, und so ein kleines oder mittleres Geschäft
 aufziehen können.
\end_layout

\begin_layout Subsubsection
Antragstellung
\end_layout

\begin_layout Standard
Die CPN vergibt Mikrokredite mit einer Laufzeit von bis zu 48 Monaten.
 Abhängig von der Art und des Umfangs der Unternehmung kann der Antragsteller
 bis zu 20.000 Dollar für sein Geschäftsvorhaben erhalten.
 Es gibt die Möglichkeiten a) einen einfachen Mikrokredit von 400 bis 10.000$
 aufzunehmen oder b) unter Aufnahme einer Hypothek, von 10.001 bis 20.000$.
\end_layout

\begin_layout Subsubsection
Anforderungen
\end_layout

\begin_layout Standard
Es bedarf, neben der Mitgliedschaft in der CPN, weitere Voraussetzungen,
 um einen Mikrokredit zu erhalten.
 Der Antragsteller muss bereits eine geschäftliche Tätigkeit verfolgen oder
 Inhaber eines unternehmerischen Gewerbes sein.
 Seine wirtsch.
 Aktivität muss er mit einem Dokument, dem sogenannten 
\begin_inset Quotes ald
\end_inset

RUC
\begin_inset Quotes ard
\end_inset

, oder anderen Unterlagen nachweisen.
 Desweiteren ist die Lohnabrechnung der letzten drei Monate gefordert.
\end_layout

\begin_layout Subsection
Termingeld (
\begin_inset Quotes ald
\end_inset

Deposito a Plazo Fijo
\begin_inset Quotes ard
\end_inset

)
\end_layout

\begin_layout Standard
Neben Hypothekendarlehen, Allgemeinen Krediten und Mikrokrediten kann bei
 der CPN Termingeld (Termineinlagen oder Termindepositen) angelegt werden.
 Termingelder sind kurz- bis mittelfristige Geldanlagen bei Kreditinstituten,
 bei denen die Laufzeit oder Kündigungsfrist mindestens einen Monat beträgt.
 
\end_layout

\begin_layout Quote
\begin_inset Quotes ald
\end_inset

Bei Termineinlagen verzichten die Einleger für eine bestimmte Zeit auf ihr
 Verfügungsrecht, um einen höheren Zins als bei Sichteinlagen zu erhalten.
 Der Zinssatz hängt von der Höhe der Einlage und von der vereinbarten Laufzeit
 bzw.
 Kündigungsfrist ab.
 Kunden legen solche Geldbeträge als Termineinlagen an, die für größere
 Zahlungsverpflichtungen (z.
 B.
 Steuerzahlungen) zu bestimmten Terminen bereitstehen müssen, für die kurzfristi
g keine Verwendung besteht oder für die günstigere Anlagemöglichkeiten an
 der Börse abgewartet werden sollen.
\begin_inset Quotes ard
\end_inset


\begin_inset Foot
status open

\begin_layout Plain Layout
Vgl.
 
\series bold
Pollert, 
\series default
Achim, 
\shape italic
Duden-Wirtschaft
\shape default
, Mannheim 2010.
\end_layout

\end_inset


\end_layout

\begin_layout Standard
Der auf dem Konto angelegte Betrag kann nicht abgehoben werden, bis die
 vertraglich abgesicherte Frist der monatlichen Verzinsung auf dem Sparbuch
 eingegangen ist oder nach Ende der Periode, nach welcher der Kreditvertrag
 abgelaufen ist.
\end_layout

\begin_layout Paragraph
Eigenschaften
\end_layout

\begin_layout Standard
Im nebenstehenden Schaubild ist eine Übersicht über die zu erhaltenden Leistunge
n, abhängig von der Laufdauer der Termineinlage, aufgeführt.
 In der linken Spalte ist die Laufdauer der Termineinlage aufgeführt, in
 der Mitte der einzuzahlende Betrag und rechts die zu erwartende Rendite.
 Für eine Termineinlage von 190 Tagen entspricht die Rendite 9% des angelegten
 Kapitals.
\end_layout

\begin_layout Standard
\begin_inset Wrap figure
lines 0
placement r
overhang 0in
width "60col%"
status open

\begin_layout Plain Layout
\begin_inset Caption

\begin_layout Plain Layout
Tabelle - 
\begin_inset Quotes ald
\end_inset

Deposito a Plazo Fijo
\begin_inset Quotes ard
\end_inset


\end_layout

\end_inset


\end_layout

\begin_layout Plain Layout
\begin_inset Graphics
	filename tablaPlazoFijio.jpg
	scale 55

\end_inset


\end_layout

\begin_layout Plain Layout
\begin_inset CommandInset href
LatexCommand href
name "www.cooperando.fin.ec/demo/index.php?option=com_content&view"
target "http://www.cooperando.fin.ec/demo/index.php?option=com_content&view=article&id=68&Itemid=71"

\end_inset


\end_layout

\begin_layout Plain Layout

\end_layout

\end_inset


\end_layout

\begin_layout Paragraph
Gewinn / Vorteile
\end_layout

\begin_layout Standard
Handhabung der besten Raten auf dem Markt und Sicherheit auf das Guthaben,
 zusätzlich zum Saldo, und Verzinseszinsung.
 Dies gilt nur, wenn die gutgeschriebenen Zinsen auf das Sparkonto, nicht
 monatlich vom Kontoinhaber abgehoben werden.
 Außerdem nimmt der Kontoinhaber automatisch, mit jeder Einzahlung auf sein
 Konto, an Werbeaktionen teil mit der Möglichkeit auf Gewinn hochwertiger
 Preise.
\end_layout

\begin_layout Section
Verlauf des Praktikums
\end_layout

\begin_layout Subsection
Kontaktaufnahme
\end_layout

\begin_layout Standard
Vor Praktikumsbeginn fand ein Kontakt mit David Hurtado statt, der mich
 während des gesamten Praktikums in Ecuador betreute und mein Vorgesetzter
 war.
 Herrn Hurtado habe ich während eines Aufenthalts in Ecuador kennengelernt
 und stehe mit ihm in regelmäßigem Kontakt.
 Er ist Chef der Marketingabteilung der CPN und koordiniert alle Arbeiten,
 die in Zusammenhang mit der Außendarstellung der CPN stehen.
 Auf Nachfrage zeigte er sich sehr offen gegenüber meiner Anfrage, ein Praktikum
 in seiner Abteilung zu leisten.
\end_layout

\begin_layout Subsection
Aufgaben, Projekte und Tätigkeiten als Praktikant
\end_layout

\begin_layout Standard
Als Praktikant des Chefs der Marketingabteilung bereitete ich Termine inhaltlich
 vor und nahm eingehende Telefongespräche entgegen.
 Außerdem arbeitete ich Herrn Hurtado zu und half ihm bei der Erstellung
 eines Marketingprospektes der dazu diente, über die Arbeit der CPN zu informier
en und die verschiedenen Leistungen der CPN in Form eines Flyers in übersichtlic
her Form darzustellen.
 Desweiteren wurde mir die Durchführung administrativer und organisatorischer
 Aufgaben wie der Terminplanung David Hurtados aufgetragen.
  
\end_layout

\begin_layout Subsection
Abteilungen
\end_layout

\begin_layout Standard
Für mein Praktikum bei der 
\begin_inset Quotes ald
\end_inset

CPN
\begin_inset Quotes ard
\end_inset

 war es mir besonders wichtig, einen möglichst umfassenden Einblick der
 Arbeit einer Polizeikooperative sowie der Arbeit innerhalb einer einzelnen
 Abteilung im speziellen zu erhalten.
 
\end_layout

\begin_layout Subsubsection
Marketingabteilung
\end_layout

\begin_layout Standard
Die Aufgabenfelder der Marketingabteilung bestehen in der eigenständigen
 Planung und Umsetzung von Marketingkonzepten.
 Im Besonderen geht es um die Bereiche 
\end_layout

\begin_layout Itemize
Erstellung/Koordination von Anzeigen und Produktbroschüren 
\end_layout

\begin_layout Itemize
Abstimmung von Presseaktivitäten mit der PR-Agentur und intern 
\end_layout

\begin_layout Itemize
Überwachen und Weiterentwickeln der Corporate Identity 
\end_layout

\begin_layout Itemize
Abteilungsübergreifende Schnittstelle für alle Kommunikationsaktivitäten
 (Online sowie Print) 
\end_layout

\begin_layout Itemize
Externe und Interne Veranstaltungsplanung und -durchführung 
\end_layout

\begin_layout Itemize
Erstellung von Marktanalysen, Wettbewerbsbeobachtung und Dokumentation 
\end_layout

\begin_layout Itemize
Betreuung der Kooperativenwebsite
\end_layout

\begin_layout Itemize
Aufbereitung von Wettbewerbsinformationen
\end_layout

\begin_layout Standard
Konkret ging es während meines Praktikums um die Neustrukturierung der CPN-Inter
netpräsenz.
 Während meiner Anwesenheit wurde das komplette Frontend der Webseite (
\begin_inset CommandInset href
LatexCommand href
name "http://www.cooperando.fin.ec"
target "http://www.cooperando.fin.ec/demo/"

\end_inset

) rundum erneuert und auf Adobe-Flash
\begin_inset script superscript

\begin_layout Plain Layout
(R)
\end_layout

\end_inset

 umgestellt.
 Dadurch soll dem Besucher eine schnellere Orientierung, bei gleichzeitiger
 Verbesserung der optischen Präsenz, geboten werden.
 Mein Aufgabenbereich bezieht sich auf das Formatieren der bereitgestellten
 Bilder für die Internetseite.
 Sie sollen auf das richtige Format gebracht werden, um so die transparenten
 Bereiche darstellen zu können.
 Darüberhinaus gehört zu meinem Aufgabenbereich bei Herrn Hurtado die Terminverw
altung, Aktenablage und die Teilnahme an verschiedenen Gesprächen und Sitzungen.
 In einer Sitzung, am 24.
 August, geht es um den direkten Kontakt zu den CPN-Mitgliedern über das
 Internet.
 Eine Mitarbeiterin bringt die Idee hervor künftig weniger Flyer drucken
 zu lassen, dafür verstärkt auf die Internetseite über Email-Newsletter
 und im Kontakt mit Bankmitgliedern aufmerksam zu machen.
 Dieser Vorschlag findet allgemeine Anerkennung.
 
\end_layout

\begin_layout Standard
In folgendem Diagramm wird die Struktur der CPN aufgezeigt.
 Das Originaldiagramm ist im Anhang dieses Praktikumsberichtes zu finden.
 Für diesen Praktikumsbericht habe ich die einzelnen Abteilungen übersetzt
 und die Grafik entsprechend angepasst.
\end_layout

\begin_layout Standard
\begin_inset Float figure
placement H
wide false
sideways false
status open

\begin_layout Plain Layout
\begin_inset Caption

\begin_layout Plain Layout
Organigramm 'CPN'
\end_layout

\end_inset


\end_layout

\begin_layout Plain Layout
\begin_inset Graphics
	filename organigramm.png
	scale 50

\end_inset


\end_layout

\begin_layout Plain Layout
\begin_inset CommandInset href
LatexCommand href
name "http://www.cooperando.fin.ec/demo/index.php?option=com_content&view=article&id=56&Itemid=56"
target "http://www.cooperando.fin.ec/demo/index.php?option=com_content&view=article&id=56&Itemid=56"

\end_inset


\end_layout

\end_inset


\end_layout

\begin_layout Standard
Das Arbeitsklima während meines Praktikums ist sehr angenehm.
 Herr Hurtado betreut mich sehr gut.
 Zu keinem Zeitpunkt meines Praktikums bekomme ich das Gefühl als Praktikant
 weniger geschätzt zu werden, als andere Kollegen und Kolleginnen.
 Ich habe das Gefühl von den Arbeitsabläufen der CPN einen umfassenden Eindruck
 zu erhalten.
\end_layout

\begin_layout Subsubsection
Weitere Abteilungen der CPN
\end_layout

\begin_layout Standard
Neben der Marketingabteilung gibt es weitere Abteilungen welche diversen
 Aufgabenbereichen zugeteilt sind.
 Neben der Marketingabteilung sind des weiteren die Abteilungen 'Human Ressource
s', 'Finanzen' sowie die 'Systemadministration' zu finden.
 Sie alle sind Unterabteilungen des Geschäftsbereichs 
\begin_inset Quotes ald
\end_inset

Strategische Operationen
\begin_inset Quotes ard
\end_inset

.
 Die Hauptaufgabe des 
\series bold
Rechnungswesens
\series default
 der Kooperative besteht darin, das gesamte Geschäftsgeschehen und insbesondere
 die Koordination der Finanzierungsprogramme zu erfassen, zu überwachen
 und auszuwerten.
 
\end_layout

\begin_layout Standard
Allgemein hat das Rechnungswesen die Aufgabe, das gesamte Unternehmensgeschehen
 zahlenmäßig zu erfassen, zu überwachen und auszuwerten.
 Im besonderen unterscheidet man: 
\end_layout

\begin_layout Enumerate

\series bold
Dokumentationsaufgabe:
\series default
 Aufzeichnung sämtlicher Geschäftsfälle anhand von Belegen: zeitnah, zeitrichtig
, geordnet, lückenlos, wahr und fortlaufend.
 
\end_layout

\begin_layout Enumerate

\series bold
Rechenschaftslegungs- und Informationsaufgabe:
\series default
 Periodenweise (jährliche) Berichterstattung an Unternehmenseigner, Behörden,
 Gläubiger (Kreditgeber) usw.
 über Vermögens- und Ertragslage (Jahresabschluss).
 
\end_layout

\begin_layout Enumerate

\series bold
Kontrollaufgabe:
\series default
 Aussagen über Produktivität, Wirtschaftlichkeit u.
 Rentabilität des Unternehmens, seiner Betriebe und anderen Teilsysteme
 (z.B.
 Kostenstellen).
 
\end_layout

\begin_layout Enumerate

\series bold
Dispositionsaufgabe:
\series default
 Bereitstellung von Zahlenmaterial als Grundlage für unternehmerische Entscheidu
ngen, z.B.
 über Investitionen, Märkte, Produktsortimente oder Absatzpolitiken.
 
\begin_inset Foot
status open

\begin_layout Plain Layout
vgl.
 
\series bold
Zingel
\series default
, Harry: 
\shape italic
Grundlagen der kaufmännischen Rechnungslegung
\shape default
, 2001-09, S.
 3.
\end_layout

\end_inset


\end_layout

\begin_layout Standard
Die 
\series bold
Finanzabteilung
\series default
 trifft Investitions- und Finanzierungsentscheidungen und sorgt dafür, die
 Zahlungsfähigkeit der Kooperative zu jedem Zeitpunkt gesichert ist.
 Im Bereich Steuern wird versucht, die Steuerbelastung der Kooperative zu
 optimieren, wobei der Handlungsspielraum in gewissem Maße durch die national
 gültigen Steuergesetze und staatliche Vorschriften eingeschränkt ist.
 Die Aufgabe des 
\series bold
Controlling
\series default
 besteht einerseits in der laufenden Kontrolle des Arbeitsablaufs und anderersei
ts in der am Periodenende abschließenden Kontrolle.
 Das Controlling arbeitet zudem zukunftsorientiert und gibt an Hand der
 ausgewerteten Daten Handlungsempfehlungen ab und stellt die optimale Informatio
nsversorgung aller sicher.
 Dies ist nötig zum Treffen von Entscheidungen.
 Eine genaue Auflistung aller Abteilungen finden Sie im unten aufgeführten
 Organigramm.
\end_layout

\begin_layout Subsubsection
Buchhaltung
\end_layout

\begin_layout Standard
Während ich mein Praktikum bei Ing.
 David Hurtado in der Marketingabteilung ableistete, hatte ich die Gelegenheit,
 mir eine weitere Abteilungen der CPN anzusehen und für einen Tag dort zu
 arbeiten.
 Dieser 
\begin_inset Quotes ald
\end_inset

Ausflug
\begin_inset Quotes ard
\end_inset

 ermöglichte mir, die Arbeit der Buchhaltung genauer kennenzulernen und
 einen tieferen Einblick in die Strukturen der CPN zu erhalten.
 Die Buchhaltung ist die Organisationseinheit der Kooperative, die die Buchführu
ng erstellt.
 
\end_layout

\begin_layout Quotation
\begin_inset Quotes ald
\end_inset

Die Buchhaltung dient primär der Dokumentation der Geschäftsvorfälle auf
 der Basis von Belegen.
 Ihr kommt eine zentrale Stellung im Rahmen des gesamten Unternehmens zu,
 denn sie liefert die gesetzlich geforderten Informationen für das externe
 und das interne Rechnungswesen.
 Im gesetzlich vorgeschriebenen Jahresabschluss [wird] über die Höhe des
 Vermögens und des Kapitals sowie des Erfolgs des Unternehmens Rechenschaft
 abgelegt
\begin_inset Quotes ard
\end_inset


\begin_inset Foot
status collapsed

\begin_layout Plain Layout
vgl.
 
\series bold
Schüler
\series default
, Mirja: 
\shape italic
Einführung in das betriebliche Rechnungswesen, Buchführung für Industrie-
 und Handelsbetriebe
\shape default
, Hamburg 2005, S.
 1.
\end_layout

\end_inset


\end_layout

\begin_layout Standard
Während meines Besuchs in der Buchhaltung durfte ich den Mitarbeitern bei
 der Bilanzierung des aktuellen Geschäftsjahres über die Schulter sehen.
 Dabei ging es um die Auflistung der von der CPN im aktuellen Geschäftsjahr
 erbrachten Geschäftsdienstleistungen.
 Desweiteren wurde ich mit der Ablage einiger Unterlagen beauftragt.
 
\begin_inset Newpage newpage
\end_inset


\end_layout

\begin_layout Section
Bewertung des Praktikums
\end_layout

\begin_layout Standard
Durch das oben beschriebene Praktikum in Ecuador konnte ich die während
 meines Studiums erlernten Wirtschaftskenntnisse anwenden und vertiefen.
 Dieser Aspekt war mir bei dem Verlauf des Praktikums von großer Bedeutung.
 Doch auch kulturell habe ich viele Erfahrungen dadurch gesammelt.
 Das Praktikum hat für mein Studium insofern einen hohen Stellenwert, da
 ich einen Einblick in die Arbeit im Marketingbereich erhalten habe.
 Insbesondere die finanzwirtschaftliche Tätigkeit der Kooperative mit der
 Einbindung im Banken- und Mikrokreditsektor waren für mein weiteres Verständnis
 wirtschaftlicher Zusammenhänge großem Mehrwert.
 Von großem Vorteil war es, die Kultur eines fremden Landes direkt kennen
 zu lernen und eigenständig Projekte zu führen und zu entwickeln.
\end_layout

\begin_layout Standard
Bezüglich der Bürotätigkeit habe ich gelernt, sowohl telefonisch als auch
 persönlich Mitglieder- und Kundengespräche in einer anderen Sprache zu
 führen.
 Wo am Anfang noch Unsicherheiten auftraten, habe ich gemerkt, dass diese
 im Laufe der Zeit immer mehr verschwanden.
 Insgesamt war das Praktikum für mich sehr gut, da ich meine Kenntnisse
 und Eigenschaften wie Anpassungsfähigkeit, oder Teamfähigkeit unter Beweis
 stellen und einen Einblick in die verschiedensten Tätigkeiten gewinnen
 konnte.
 Die wichtigste Erkenntnis liegt jedoch darin, dass mir durch das Praktikum
 die spätere Entscheidung, im finanzwirtschaftlichen Sektor beruflich zu
 betätigen, konkretisiert wurde.
 Generell wurde mir ein guter Einblick in den Wirtschaftsbereich verschafft,
 in dem ich mir durchaus vorstellen könnte zu arbeiten.
 Die Betreuung war sehr gut, da mir sukzessive Verantwortung übertragen
 wurde und ich immer die Möglichkeit hatte, neue Tätigkeiten zu erlernen
 und vor allem eigenständig Projekte zu entwickeln und führen.
 
\end_layout

\begin_layout Standard
Das Praktikum bei der 
\begin_inset Quotes ald
\end_inset

Cooperativa de la Policía Nacional
\begin_inset Quotes ard
\end_inset

 in Ecuador ist geeignet, um Kenntnisse im Bereich der Bankentätigkeit zu
 sammeln.
 Neben klassischen Bankdienstleistungen bietet die 
\begin_inset Quotes ald
\end_inset

Cooperativa
\begin_inset Quotes ard
\end_inset

 darüberhinaus Mikrokredite und Darlehensprogramme, welche den Finanzprogrammen
 der Privatbanken entspricht, in ihrem Umfang jedoch weniger komplex und
 leichter verständlich sind.
 Dies erleichterte mir den Einblick in die Finanzwelt und die Praktiken
 der Kreditvergabe.
 Der Vorteil besteht in der weiten Streuung der Tätigkeiten, in der sich
 die CPN engagiert.
 Dies erleichterte mir den Einblick in die Finanzwelt und die Arbeit einer
 Kooperative, die aus einem Zusammenschluss vieler Mitarbeiter der Polizei
 entstand.
 Die zu Anfang des Praktikums aufkommenden Unsicherheiten meiner Kollegen
 und Kolleginnen bezüglich meiner Spanischkenntnisse verschwanden schnell
 und wich einem konstruktiven Arbeitsverhältnis, bei dem beide Seiten profitiert
en.
 Rückblickend gesehen ist ein Praktikum bei einem Unternehmen, außerhalb
 des eigenen Landes, eine große Gelegenheit, da sich so die Verständigung
 auf die grundlegenden Arbeitsvorgänge bezieht und die eigenen sozialen
 Kompetenzen in Form gemeinsamer Projekte geschult werden.
 Insgesamt war mein Praktikumsaufenthalt in er CPN eine sehr lohnende Erfahrung,
 die mir auch hinsichtlich meines Studiums nützliche Kenntnisse einbringt.
 Beigefügt ist eine Kopie der Praktikumsbescheinigung.
\end_layout

\begin_layout Standard
\begin_inset VSpace bigskip
\end_inset


\end_layout

\begin_layout Standard
\begin_inset VSpace bigskip
\end_inset


\end_layout

\begin_layout Standard
\begin_inset VSpace bigskip
\end_inset


\end_layout

\begin_layout Standard
\begin_inset VSpace bigskip
\end_inset


\end_layout

\begin_layout Standard
\begin_inset VSpace bigskip
\end_inset


\end_layout

\begin_layout Standard
\begin_inset VSpace bigskip
\end_inset


\end_layout

\begin_layout Standard
\begin_inset VSpace bigskip
\end_inset


\end_layout

\begin_layout Standard
\begin_inset VSpace bigskip
\end_inset


\end_layout

\begin_layout Standard
\begin_inset VSpace bigskip
\end_inset


\end_layout

\begin_layout Standard
\begin_inset VSpace bigskip
\end_inset


\end_layout

\begin_layout Right Address
Gelesen und für gut befunden,
\end_layout

\begin_layout Standard
\begin_inset Newpage newpage
\end_inset


\end_layout

\begin_layout Section
Anhang
\end_layout

\begin_layout Section
\start_of_appendix
Schaubilder und Grafiken
\end_layout

\begin_layout Standard
\begin_inset Graphics
	filename organigramm.png
	width 18cm

\end_inset


\end_layout

\begin_layout Description
\begin_inset Graphics
	filename directorio.jpg
	width 18cm

\end_inset


\end_layout

\begin_layout Description
\begin_inset Graphics
	filename tablaPlazoFijio.jpg
	width 9cm

\end_inset


\end_layout

\begin_layout Description
\begin_inset Graphics
	filename hipotecario.jpg

\end_inset


\end_layout

\begin_layout Standard
\begin_inset Newpage newpage
\end_inset


\end_layout

\begin_layout Section
Praktikumsbescheinigung
\end_layout

\begin_layout Standard

\end_layout

\end_body
\end_document
