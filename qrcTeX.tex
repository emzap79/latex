% Reference Card for TeX
% Copyright (c) 1992 Joseph H. Silverman. May be freely distributed.
% Created Tuesday, August 18, 1992
% Thanks to Stephen Gildea for the multicolumn macro package
% which I modified from his GNU emacs reference card
% TeX is a trademark of the American Mathematical Society

%**start of header
\newcount\columnsperpage

% This file can be printed with 1, 2, or 3 columns per page (see below).
% [For 2 or 3 columns, you'll need 6 and 8 point fonts.]
% Specify how many you want here.  Nothing else needs to be changed.

\columnsperpage=1

% Smaller (97%) pdf file with horizontal offset 1.5in
% dvipdfm -l -m 0.97 -x 1.5in -o TeXRefCard.v1.5.pdf TeXRefCard.v1.5.dvi

% There are a couple extra sections included at the end of the document
% (after the \bye) that didn't fit into six columns.
% It would be nice to have additional sections covering:
% \hrules and \vrules, Registers
% \input and \output files (including \read, \write, \message)

% This reference card is distributed in the hope that it will be useful,
% but WITHOUT ANY WARRANTY; without even the implied warranty of
% MERCHANTABILITY or FITNESS FOR A PARTICULAR PURPOSE.

% This file is intended to be processed by plain TeX (TeX82).
%
% The final reference card has six columns, three on each side.
% This file can be used to produce it in any of three ways:
% 1 column per page
%    produces six separate pages, each of which needs to be reduced to 80%.
%    This gives the best resolution.
% 2 columns per page
%    produces three already-reduced pages.
%    You will still need to cut and paste.
% 3 columns per page
%    produces two pages which must be printed sideways to make a
%    ready-to-use 8.5 x 11 inch reference card.
%    For this you need a dvi device driver that can print sideways.
% Which mode to use is controlled by setting \columnsperpage above.
%
% Author:
%  Joseph H. Silverman
%  Brown University Mathematics Department
%  Providence, RI 02912 USA
%  Internet:  jhs@math.brown.edu
%  (reference card macros due to Stephen Gildea)

% History:
%  Version 1.0 - August 1992, distributed at Brown University
%  Version 1.1 - April 1994, general distribution
%  Version 1.2 - January 1996, minor corrections and changes
%  Version 1.3 - November 1998, minor corrections and changes
%  Version 1.4 - January 2003, minor corrections and changes
%  Version 1.5 - January 2007, minor corrections and changes

\def\versionnumber{1.5}  % Version of this reference card
\def\year{2007}
\def\month{January}
\def\version{\month\ \year\ v\versionnumber}


\def\shortcopyrightnotice{
\vskip 0pt plus 2 fill\begingroup\parskip=0pt\small
   \centerline{\copyright\ \number\year\ J.H. Silverman,
   Permissions on back.  v\versionnumber}

Send comments and corrections to J.H. Silverman, Math.\ Dept.,
Brown Univ., Providence, RI 02912 USA.
$\langle$jhs@math.brown.edu$\rangle$

\endgroup}

\def\copyrightnotice{
\vskip 1ex plus 2 fill\begingroup\parskip=0pt\small
\centerline{Copyright \copyright\ \year\ J.H. Silverman, \version}
\centerline{Math.\ Dept., Brown Univ., Providence, RI 02912 USA}
\centerline{\TeX\ is a trademark of the American Mathematical Society}

Permission is granted to make and distribute copies of
this card provided the copyright notice and this permission notice
are preserved on all copies.

\endgroup}

% make \bye not \outer so that the \def\bye in the \else clause below
% can be scanned without complaint.
\def\bye{\par\vfill\supereject\end}

\newdimen\intercolumnskip
\newbox\columna
\newbox\columnb

\def\ncolumns{\the\columnsperpage}

\message{[\ncolumns\space
   column\if 1\ncolumns\else s\fi\space per page]}

\def\scaledmag#1{ scaled \magstep #1}

% This multi-way format was designed by Stephen Gildea
% October 1986.
\if 1\ncolumns
   \hsize 4in
   \vsize 10in
   \voffset -.7in
   \font\titlefont=\fontname\tenbf \scaledmag3
   \font\headingfont=\fontname\tenbf \scaledmag2
   \font\smallfont=\fontname\sevenrm
   \font\smallsy=\fontname\sevensy

   \footline{\hss\folio}
   \def\makefootline{\baselineskip10pt\hsize6.5in\line{\the\footline}}
\else
   \hsize 3.2in
   \vsize 7.95in
   \hoffset -.75in
   \voffset -.745in
   \font\titlefont=cmbx10 \scaledmag2
   \font\headingfont=cmbx10 \scaledmag1
   \font\smallfont=cmr6
   \font\smallsy=cmsy6
   \font\eightrm=cmr8
   \font\eighti=cmmi8
   \font\eightsy=cmsy8
   \font\eightbf=cmbx8
   \font\eighttt=cmtt8
   \font\eightit=cmti8
   \font\eightsl=cmsl8
   \textfont0=\eightrm
   \textfont1=\eighti
   \textfont2=\eightsy
   \def\rm{\eightrm}
   \def\bf{\eightbf}
   \def\tt{\eighttt}
   \def\it{\eightit}
   \def\sl{\eightsl}
   \normalbaselineskip=.8\normalbaselineskip
   \normallineskip=.8\normallineskip
   \normallineskiplimit=.8\normallineskiplimit
   \normalbaselines\rm          %make definitions take effect

   \if 2\ncolumns
     \let\maxcolumn=b
     \footline{\hss\rm\folio\hss}
     \def\makefootline{\vskip 2in \hsize=6.86in\line{\the\footline}}
   \else \if 3\ncolumns
     \let\maxcolumn=c
     \nopagenumbers
   \else
     \errhelp{You must set \columnsperpage equal to 1, 2, or 3.}
     \errmessage{Illegal number of columns per page}
   \fi\fi

   \intercolumnskip=.46in
   \def\abc{a}
   \output={%
       % This next line is useful when designing the layout.
       %\immediate\write16{Column \folio\abc\space starts with \firstmark}
       \if \maxcolumn\abc \multicolumnformat \global\def\abc{a}
       \else\if a\abc
        \global\setbox\columna\columnbox \global\def\abc{b}
         %% in case we never use \columnb (two-column mode)
         \global\setbox\columnb\hbox to -\intercolumnskip{}
       \else
        \global\setbox\columnb\columnbox \global\def\abc{c}\fi\fi}
   \def\multicolumnformat{\shipout\vbox{\makeheadline
       \hbox{\box\columna\hskip\intercolumnskip
         \box\columnb\hskip\intercolumnskip\columnbox}
       \makefootline}\advancepageno}
   \def\columnbox{\leftline{\pagebody}}

   \def\bye{\par\vfill\supereject
     \if a\abc \else\null\vfill\eject\fi
     \if a\abc \else\null\vfill\eject\fi
     \end}
\fi

% ***** Verbatim typesetting *****

% Verbatim typesetting is done by
%    \verbatim"stuff to verbatim typeset"
% Any character can be used in place of ".
% E.g. \verbatim?stuff? or \verbatim|stuff|.  Cf. TeXbook pp.380-382

\def\uncatcodespecials{\def\do##1{\catcode`##1=12}\dospecials}
\def\setupverbatim{\tt%
\def\par{\leavevmode\endgraf}\catcode`\`=\active%
\obeylines\uncatcodespecials\obeyspaces}
\def\verbatim{\begingroup\setupverbatim\doverbatim}
\def\doverbatim#1{\def\next##1#1{##1\endgroup}\next}

\def\\{\verbatim}
\def\ds{\displaystyle}
\def\SPC{\quad} % space between symbol and command

\parindent 0pt
\parskip 1ex plus .5ex minus .5ex

\def\small{\smallfont\textfont2=\smallsy\baselineskip=.8\baselineskip}

\outer\def\newcolumn{\vfill\eject}

\outer\def\title#1{{\titlefont\centerline{#1}}\vskip 1ex plus .5ex minus.5ex}

%\outer\def\section#1{\par\filbreak
%  \vskip 1ex plus 2ex minus 2ex {\headingfont #1}\mark{#1}%
%  \vskip 1ex plus 1ex minus .5ex}
\outer\def\section#1{\par\filbreak
   \vskip .75ex plus 1ex minus 2ex {\headingfont #1}\mark{#1}%
   \vskip .5ex plus .5ex minus .5ex}


\def\paralign{\vskip\parskip\halign}

\def\<#1>{$\langle${\rm #1}$\rangle$}

\def\begintext{\par\leavevmode\begingroup\parskip0pt\rm}
\def\endtext{\endgroup}


% ************  TEXT STARTS HERE **************************

\title{\TeX~Reference Card}

\centerline{(for Plain \TeX)}

\section{Greek Letters}

% ***** Three Column Format *****
\paralign to\hsize{%
$#$\hfil\qquad\tabskip=0pt&#\hfil\tabskip=0pt plus 1 fil
&$#$\hfil\SPC\tabskip=0pt&#\hfil\tabskip=0pt plus 1 fil
&$#$\hfil\SPC\tabskip=0pt&#\hfil\cr
%----------- 3 Column Data -------------------
\alpha&\\"\alpha"&\iota&\\"\iota"&\varrho&\\"\varrho"\cr
\beta&\\"\beta"&\kappa&\\"\kappa"&\sigma&\\"\sigma"\cr
\gamma&\\"\gamma"&\lambda&\\"\lambda"&\varsigma&\\"\varsigma"\cr
\delta&\\"\delta"&\mu&\\"\mu"&\tau&\\"\tau"\cr
\epsilon&\\"\epsilon"&\nu&\\"\nu"&\upsilon&\\"\upsilon"\cr
\varepsilon&\\"\varepsilon"&\xi&\\"\xi"&\phi&\\"\phi"\cr
\zeta&\\"\zeta"&o&\\"\o"&\varphi&\\"\varphi"\cr
\eta&\\"\eta"&\pi&\\"\pi"&\chi&\\"\chi"\cr
\theta&\\"\theta"&\varpi&\\"\varpi"&\psi&\\"\psi"\cr
\vartheta&\\"\vartheta"&\rho&\\"\rho"&\omega&\\"\omega"\cr
\noalign{\vskip\parskip}
\Gamma&\\"\Gamma"&\Xi&\\"\Xi"&\Phi&\\"\Phi"\cr
\Delta&\\"\Delta"&\Pi&\\"\Pi"&\Psi&\\"\Psi"\cr
\Theta&\\"\Theta"&\Sigma&\\"\Sigma"&\Omega&\\"\Omega"\cr
\Lambda&\\"\Lambda"&\Upsilon&\\"\Upsilon"\cr
}

\section{Symbols of Type Ord}

% ***** Three Column Format *****
\paralign to\hsize{%
$#$\hfil\SPC\tabskip=0pt&#\hfil\tabskip=0pt plus 1 fil
&$#$\hfil\SPC\tabskip=0pt&#\hfil\tabskip=0pt plus 1 fil
&$#$\hfil\SPC\tabskip=0pt&#\hfil\cr
%----------- 3 Column Data -------------------
\aleph&\\"\aleph"&\prime&\\"\prime"&\forall&\\"\forall"\cr
\hbar&\\"\hbar"&\emptyset&\\"\emptyset"&\exists&\\"\exists"\cr
\imath&\\"\imath"&\nabla&\\"\nabla"&\neg&\\"\neg" or \\"\lnot"\cr
\jmath&\\"\jmath"&\surd&\\"\surd"&\flat&\\"\flat"\cr
\ell&\\"\ell"&\top&\\"\top"&\natural&\\"\natural"\cr
\wp&\\"\wp"&\bot&\\"\bot"&\sharp&\\"\sharp"\cr
\Re&\\"\Re"&\|&\\"\|"&\clubsuit&\\"\clubsuit"\cr
\Im&\\"\Im"&\angle&\\"\angle"&\diamondsuit&\\"\diamondsuit"\cr
\partial&\\"\partial"&\triangle&\\"\triangle"&\heartsuit&\\"\heartsuit"\cr
\infty&\\"\infty"&\backslash&\\"\backslash"&\spadesuit&\\"\spadesuit"\cr
}

\section{Large Operators}

% ***** Three Column Format *****
\paralign to\hsize{%
$#$\hfil\SPC\tabskip=0pt&#\hfil\tabskip=0pt plus 1 fil
&$#$\hfil\SPC\tabskip=0pt&#\hfil\tabskip=0pt plus 1 fil
&$#$\hfil\SPC\tabskip=0pt&#\hfil\cr
%----------- 3 Column Data -------------------
\sum &\\"\sum"&\bigcap &\\"\bigcap"
      &\bigodot &\\"\bigodot"\cr
\prod &\\"\prod"&\bigcup &\\"\bigcup"
      &\bigotimes &\\"\bigotimes"\cr
\coprod &\\"\coprod"&\bigsqcup
      &\\"\bigsqcup"&\bigoplus &\\"\bigoplus"\cr
\int &\\"\int"&\bigvee &\\"\bigvee"
      &\biguplus &\\"\biguplus"\cr
\oint &\\"\oint"&\bigwedge &\\"\bigwedge"\cr
}

\section{Binary Operations}

% ***** Three Column Format *****
\paralign to\hsize{%
$#$\hfil\SPC\tabskip=0pt&#\hfil\tabskip=0pt plus 1 fil
&$#$\hfil\SPC\tabskip=0pt&#\hfil\tabskip=0pt plus 1 fil
&$#$\hfil\SPC\tabskip=0pt&#\hfil\cr
%----------- 3 Column Data -------------------
\pm&\\"\pm"&\cap&\\"\cap"&\vee&\\"\vee" or \\"\lor"\cr
\mp&\\"\mp"&\cup&\\"\cup"&\wedge&\\"\wedge" or \\"\land"\hidewidth\cr
\setminus&\\"\setminus"&\uplus&\\"\uplus"&\oplus&\\"\oplus"\cr
\cdot&\\"\cdot"&\sqcap&\\"\sqcap"&\ominus&\\"\ominus"\cr
\times&\\"\times"&\sqcup&\\"\sqcup"&\otimes&\\"\otimes"\cr
\ast&\\"\ast"&\triangleleft&\\"\triangleleft"&\oslash&\\"\oslash"\cr
\star&\\"\star"&\triangleright&\\"\triangleright"&\odot&\\"\odot"\cr
\diamond&\\"\diamond"&\wr&\\"\wr"&\dagger&\\"\dagger"\cr
\circ&\\"\circ"&\bigcirc&\\"\bigcirc"&\ddagger&\\"\ddagger"\cr
\bullet&\\"\bullet"&\bigtriangleup&\\"\bigtriangleup"&\amalg&\\"\amalg"\cr
\div&\\"\div"&\bigtriangledown&\\"\bigtriangledown"\cr
}

\section{Page Layout}

\halign{#\hfil\qquad&#\hfil\cr
\\"\hsize="\<dimen>&set width of page\cr
\\"\vsize="\<dimen>&set height of page\cr
\\"\displaywidth="\<dimen>&set width of math displays\cr
\\"\hoffset="\<dimen>&move page horizontally\cr
\\"\voffset="\<dimen>&move page vertically\cr
}

\section{Relations}
% ***** Three Column Format *****
\paralign to\hsize{%
$#$\hfil\SPC\tabskip=0pt&#\hfil\tabskip=0pt plus 1 fil
&$#$\hfil\SPC\tabskip=0pt&#\hfil\tabskip=0pt plus 1 fil
&$#$\hfil\SPC\tabskip=0pt&#\hfil\cr
%----------- 3 Column Data -------------------
\leq&\\"\leq" or \\"\le"&\geq&\\"\geq" or \\"\ge"&\equiv&\\"\equiv"\cr
\prec&\\"\prec"&\succ&\\"\succ"&\sim&\\"\sim"\cr
\preceq&\\"\preceq"&\succeq&\\"\succeq"&\simeq&\\"\simeq"\cr
\ll&\\"\ll"&\gg&\\"\gg"&\asymp&\\"\asymp"\cr
\subset&\\"\subset"&\supset&\\"\supset"&\approx&\\"\approx"\cr
\subseteq&\\"\subseteq"&\supseteq&\\"\supseteq"&\cong&\\"\cong"\cr
\sqsubseteq&\\"\sqsubseteq"&\sqsupseteq&\\"\sqsupseteq"&\bowtie&\\"\bowtie"\cr
\in&\\"\in"&\notin&\\"\notin"&\ni&\\"\ni" or \\"\owns"\cr
\vdash&\\"\vdash"&\dashv&\\"\dashv"&\models&\\"\models"\cr
\smile&\\"\smile"&\mid&\\"\mid"&\doteq&\\"\doteq"\cr
\frown&\\"\frown"&\parallel&\\"\parallel"&\perp&\\"\perp"\cr
\propto&\\"\propto"\cr
\noalign{\vskip\parskip
\hbox{Most relations can be negated by prefixing them with \\"\not".}
\vskip\parskip}
\not\equiv&\\"\not\equiv"&\notin&\\"\notin"&\ne&\\"\ne"\cr
}


\section{Arrows}

% ***** Two Column Format *****
\paralign to\hsize{%
$#$\hfil\SPC\tabskip=0pt&#\hfil\tabskip=0pt plus 1 fil
&$#$\hfil\SPC\tabskip=0pt&#\hfil\cr
%----------- 2 Column Data -------------------
\leftarrow&\\"\leftarrow" or \\"\gets"&\longleftarrow&\\"\longleftarrow"\cr
\Leftarrow&\\"\Leftarrow"&\Longleftarrow&\\"\Longleftarrow"\cr
\rightarrow&\\"\rightarrow" or \\"\to"&\longrightarrow&\\"\longrightarrow"\cr
\Rightarrow&\\"\Rightarrow"&\Longrightarrow&\\"\Longrightarrow"\cr
\leftrightarrow&\\"\leftrightarrow"&\longleftrightarrow
      &\\"\longleftrightarrow"\cr
\Leftrightarrow&\\"\Leftrightarrow"&\Longleftrightarrow
      &\\"\Longleftrightarrow"\cr
\mapsto&\\"\mapsto"&\longmapsto&\\"\longmapsto"\cr
\hookleftarrow&\\"\hookleftarrow"&\hookrightarrow&\\"\hookrightarrow"\cr
\uparrow&\\"\uparrow"&\Uparrow&\\"\Uparrow"\cr
\downarrow&\\"\downarrow"&\Downarrow&\\"\Downarrow"\cr
\updownarrow&\\"\updownarrow"&\Updownarrow&\\"\Updownarrow"\cr
\nearrow&\\"\nearrow"&\searrow&\\"\searrow"\cr
\nwarrow&\\"\nwarrow"&\swarrow&\\"\swarrow"\cr
}

\begingroup
\parskip=0pt
\begintext
The \\"\buildrel" macro puts one symbol over another. The format
is \\"\buildrel"\<superscript>\\"\over"\<relation>.
\endtext

\tabskip=\centering
\paralign to\hsize{%
\hfil$#$\hfil&\qquad#\hfil\cr
\buildrel\alpha\beta\over\longrightarrow
   &\\"\buildrel\alpha\beta\over\longrightarrow"\cr
f(x)\; {\buildrel\hbox{\smallfont def}\over=} \;x+1
   &\\"f(x)\; {\buildrel\rm def\over=} \;x+1"\cr
}
\endgroup

\section{Delimiters}

% ***** Three Column Format *****
\paralign to\hsize{%
$#$\hfil\SPC\tabskip=0pt&#\hfil\tabskip=0pt plus 1 fil
&$#$\hfil\SPC\tabskip=0pt&#\hfil\tabskip=0pt plus 1 fil
&$#$\hfil\SPC\tabskip=0pt&#\hfil\cr
%----------- 3 Column Data -------------------
[&\\"\lbrack" or \\"["&\{&\\"\lbrace" or \\"\{"&\langle&\\"\langle"\cr
]&\\"\rbrack" or \\"]"&\}&\\"\rbrace" or \\"\}"&\rangle&\\"\rangle"\cr
\vert&\\"\vert" or \\"|"&\lfloor&\\"\lfloor"&\lceil&\\"\lceil"\cr
\|&\\"\Vert" or \\"\|"&\rfloor&\\"\rfloor"&\rceil&\\"\rceil"\cr
[\![&\\"[\!["&(\!(&\\"(\!("&\langle\!\langle&\\"\langle\!\langle"\cr
]\!]&\\"]\!]"&)\!)&\\")\!)"&\rangle\!\rangle&\\"\rangle\!\rangle"\cr
}

\begintext
Left and right delimiters will be enlarged if they are prefixed with
\\"\left" or \\"\right". Each \\"\left" must have a matching \\"\right", one of
which may be an empty delimiter (\\"\left." or \\"\right.").
To specify a particular size, use the following:
\endtext

\begingroup
\parskip=0pt
\tabskip=\centering
\paralign to\hsize{%
&\hfil#\hfil\cr
\\"\bigl", \\"\bigr"&\\"\Bigl", \\"\Bigr"&\\"\biggl", \\"\biggr"\cr
}
\begintext
You can also say \\"\bigm" for a  large delimiter in the middle of a
formula, or just \\"\big" for one that acts as an ordinary symbol.
\endtext
\endgroup

\section{Every Time Insertions}
\halign{#\hfil\qquad&#\hfil\cr
\\"\everypar"&insert whenever a paragraph begins\cr
\\"\everymath"&insert whenever math in text begins\cr
\\"\everydisplay"&insert whenever displayed math begins\cr
\\"\everycr"&insert after every \\"\cr"\cr
}



\section{Accents}

% ***** Four Column Format *****
\paralign to\hsize{%
\tabskip=\centering
#\hfil & \hfil$#$\hfil & #\hfil & \hfil#\hfil\tabskip=0pt\cr
%----------- 4 Column Data -------------------
Type & \omit\hfil Example\hfil & In Math & In Text \cr
hat & \hat a & \\"\hat" & \\"\^" \cr
expanding hat & \widehat{abc} & \\"\widehat" & none\cr
check & \check a & \\"\check" & \\"\v"\cr
tilde & \tilde a & \\"\tilde" & \\"\~" \cr
expanding tilde & \widetilde{abc} & \\"\widetilde" & none\cr
acute & \acute a & \\"\acute" & \\"\'" \cr
grave & \grave a & \\"\grave" & \tt\\"\"\char`\` \cr
dot & \dot a & \\"\dot" & \\"\." \cr
double dot & \ddot a & \\"\ddot" & \\?\"? \cr
breve & \breve a & \\"\breve" & \\"\u" \cr
bar & \bar a & \\"\bar" & \\"\=" \cr
vector & \vec a & \\"\vec" & none\cr
}

\begintext
The \\"\skew"\<number> command shifts accents for proper positioning,
the larger the \<number>, the more right the shift. Compare
\endtext

% ***** Centered Text Format *****
\vbox{\tabskip=\centering
\paralign to\hsize{\hfil#\hfil\cr
%----------- Centered Text Data -------------------
\\"\hat{\hat A}" gives $\hat{\hat A}$,\qquad
\\"\skew6\hat{\hat A}" gives $\skew6\hat{\hat A}$.\cr
} }

\section{Elementary Math Control Sequences}

% ***** Other Three Column Format *****
\paralign to\hsize{%
\tabskip=\centering
#\hfil & \hfil$\ds#$\hfil & #\hfil \tabskip=0pt\cr
%----------- 3 Column Data -------------------
overline a formula&\overline{x+y}&\\"\overline{x+y}"\cr
underline a formula&\underline{x+y}&\\"\underline{x+y}"\cr
square root&\sqrt{x+2}&\\"\sqrt{x+2}"\cr
higher order roots&\root n\of {x+2}&\\"\root n\of{x+2}"\cr
fraction&{n+1\over 3}&\\"{n+1\over 3}"\cr
fraction, no line&{n+1\atop 3}&\\"{n+1\atop 3}"\cr
binomial coeff.&{n+1\choose 3}&\\"{n+1\choose 3}"\cr
braced fraction&{n+1\brace 3}&\\"{n+1\brace 3}"\cr
bracketed fraction&{n+1\brack 3}&\\"{n+1\brack 3}"\cr
}

\begintext
The following specify a style for typesetting formulas.
\endtext

% ***** Centered Text Format *****
\vbox{\tabskip=\centering
\paralign to\hsize{\hfil#\hfil\cr
%----------- Centered Text Data -------------------
\\"\displaystyle"
\\"\textstyle"
\\"\scriptstyle"
\\"\scriptscriptstyle"\cr
} }


\section{Non-Italic Function Names}

\def\lim{\mathop{\hbox{\rm lim}}}
\def\log{\mathop{\hbox{\rm log}}\nolimits}
\def\pmod#1{\allowbreak\mkern18mu(\hbox{\rm mod}\,\,#1)}
\def\bmod{\mskip-\medmuskip\mkern5mu\mathbin{\hbox{\rm mod}}\penalty900%
\mkern5mu\mskip-\medmuskip}

\paralign to\hsize{#\hfil\tabskip=\centering
&#\hfil&#\hfil&#\hfil&#\hfil&#\hfil&#\hfil&#\hfil\tabskip=0pt\cr
\\"\arccos"&\\"\cos"&\\"\csc"&\\"\exp"&\\"\ker"&\\"\limsup"&\\"\min"&\\"\sinh"\cr
\\"\arcsin"&\\"\cosh"&\\"\deg"&\\"\gcd"&\\"\lg"&\\"\ln"&\\"\Pr"&\\"\sup"\cr
\\"\arctan"&\\"\cot"&\\"\det"&\\"\hom"&\\"\lim"&\\"\log"&\\"\sec"&\\"\tan"\cr
\\"\arg"&\\"\coth"&\\"\dim"&\\"\inf"&\\"\liminf"&\\"\max"&\\"\sin"&\\"\tanh"\cr
}
\vskip-.5\baselineskip
\paralign to\hsize{#\hfil\tabskip=0pt plus 1fil&#\hfil&#\hfil\tabskip=0pt\cr
\\"a \pmod{m}"&$a\pmod{m}$&mod with parentheses\cr
\\"a \bmod m"&$a\bmod m$&mod without parentheses\cr
}
\begintext
The following examples  use \\"\mathop" to create function names.
\endtext

\vskip-.75\baselineskip
\begingroup
\openup.5\jot
\paralign to\hsize{#\hfil\tabskip=0pt plus 1fil&#\hfil&#\hfil\tabskip=0pt\cr
Example&Command&Plain \TeX\ Definition\hfil\cr
\noalign{\vskip-1\jot}
$\displaystyle \lim_{x\to2}$ & \\"\lim_{x\to2}" &
   \\"\def\lim{\mathop{\rm lim}}"   \cr
$\displaystyle \log_2$ & \multispan2 \\"\log_2" \hfill
   \\"\def\log{\mathop{\rm log}\nolimits}" \cr
}
\endgroup

\section{Footnotes, Insertions, and Underlines}

\paralign{#\hfil\quad&#\hfil\cr
\\"\footnote"\<marker>\\"{"\<text>\\"}"&footnote\cr
\\"\topinsert"\<vmode material>\\"\endinsert"&insert at top of page\cr
\\"\pageinsert"\<vmode material>\\"\endinsert"&insert on full page\cr
\\"\midinsert"\<vmode material>\\"\endinsert"&insert middle of page\cr
%\noalign{\vskip2\jot}
\\"\underbar{"\<text>\\"}"&underline text\cr
}


\shortcopyrightnotice


\section{Useful Parameters and Conversions}

\halign{#\hfil\qquad&#\hfil\cr
\\"\day,\month,\year"&the current day, month, year\cr
\\"\jobname"&name of current job\cr
\\"\romannumeral"\<number>&convert to lower case roman nums.\cr
\\"\uppercase{"\<token list>\\"}"&convert to upper case\cr
\\"\lowercase{"\<token list>\\"}"&convert to lower case\cr
}

\section{Fills, Leaders and Ellipses}

\paralign to\hsize{%
#\quad\hfil&
#\hfil\enspace\tabskip=0pt&#\hfil\tabskip=1em plus1fil&
#\hfil\enspace\tabskip=0pt&#\hfil\tabskip=1em plus1fil&
#\hfil\enspace\tabskip=0pt&#\hfil\tabskip=1em plus1fil&
#\hfil\enspace\tabskip=0pt&#\hfil\cr
\multispan3\ignorespaces Text or Math:\hfil&\dots&\\"\dots"\cr
Math:&
$\ldots$&\\"\ldots"&
$\cdots$&\\"\cdots"&
$\smash{\vdots}$&\\"\vdots"&
$\smash{\ddots}$&\\"\ddots"\cr
}

\begintext
The following fill space with the indicated item.
\endtext
\halign to\hsize{\hfil#\hfil\tabskip=\centering
&\hfil#\hfil\tabskip=\centering
&\hfil#\hfil\tabskip=\centering
&\hfil#\hfil\tabskip=0pt\cr
\\"\hrulefill"&\\"\rightarrowfill"&\\"\leftarrowfill"&\\"\dotfill"\cr
}
The general format for constructing leaders is
\halign{#\hfil\quad&#\hfil\cr
\\"\leaders"\<box or rule>\\"\hskip"\<glue>&repeat box or rule\cr
\\"\leaders"\<box or rule>\\"\hfill"&fill space with box or rule\cr
}

\section{\TeX~Fonts and Magnification}

\paralign{&#\quad\hfil&#\qquad\hfil\cr
\\"\rm"&Roman&
\\"\bf"&\bf Bold&
\\"\tt"&\tt Typewriter\cr
\\"\sl"&\sl Slant&
\\"\it"&\it Italic&
\\"\/"&``italic correction''
\cr
}

\paralign{#\hfil\qquad&#\hfil\cr
\\"\magnification="\<number>&scale document by $n/1000$\cr
\\"\magstep"\<number>&scaling factor of $1.2^n\times1000$\cr
\\"\magstephalf"&scaling factor of $\sqrt{1.2}$\cr
\\"\font\FN="\<fontname>&load a font, naming it \\"\FN"\cr
\\"\font\FN="\<fontname> {\tt at} \<dimen>
   \hglue3.6em  scaled to dimension\hidewidth\cr
%%   &load font scaled to dimension\cr
\\"\font\FN="\<fontname> {\tt scaled} \<number>
   \quad  scaled by $n/1000$\hidewidth\cr
%%   &load font scaled by $n/1000$\cr
{\tt true} \<dimen>&dimension with no scaling\cr
\\"\char"{\tt\char`\`}\\"\"$c$&print the character or symbol $c$ \cr
}

\section{Alignment Displays}

\paralign{#\hfil\qquad&#\hfil\cr
\\"\settabs"\<number>\\"\columns"&
   set equally spaced tabs\cr
\\"\settabs\+"\<sample line>\\"\cr"&
   set tabs as per sample line\cr
\\"\+"\<text$_1$>{\tt\&}\<text$_2$>{\tt\&}$\cdots$\\"\cr"
   &tabbed text to be typeset\cr
\\"\halign"&horizontal alignment\cr
\\"\halign to"\<dimen>&horizontal alignment\cr
\\"\openup"\<dimen>&add space between lines\cr
\\"\noalign{"\<vmode material>\\"}"&insert material after any \\"\cr"\cr
\\"\tabskip="\<glue>&set glue at tab stops\cr
\\"\omit"&omit the template for a column\cr
\\"\span"&span two columns\cr
\\"\multispan"\<number>&span several columns\cr
\\"\hidewidth"&ignore the width of an entry\cr
\\"\crcr"&insert \\"\cr" if one is not present\cr
}

\section{Boxes}

\halign{#\hfil\qquad&#\hfil\cr
\\"\hbox to"\<dimen>&hbox of given dimension\cr
\\"\vbox to"\<dimen>&vbox, bottom justified\cr
\\"\vtop to"\<dimen>&vbox, top justified\cr
\\"\vcenter to"\<dimen>&vbox, center justified (math only)\cr
\\"\rlap"&right overlap material\cr
\\"\llap"&left overlap material\cr
}

\section{Overfull Boxes}

\halign{#\hfil\qquad&#\hfil\cr
\\"\hfuzz"&allowable excess in hboxes\cr
\\"\vfuzz"&allowable excess in vboxes\cr
\\"\overfullrule"&width of overfull box marker. To eliminate\cr
   &\qquad entirely, set \\"\overfullrule=0pt".\cr
}

\section{Indentation and Itemized Lists}

\halign{#\hfil\qquad&#\hfil\cr
\\"\indent"&indent\cr
\\"\noindent"&do not indent\cr
\\"\parindent="\<dimen>&set indentation of paragraphs\cr
\\"\displayindent="\<dimen>&set indentation of math displays\cr
\\"\leftskip="\<dimen>&skip space on left\cr
\\"\rightskip="\<dimen>&skip space on right\cr
\\"\narrower"&make paragraph narrower\cr
\\"\item{"\<label>\\"}"&singly indented itemized list\cr
\\"\itemitem{"\<label>\\"}"&doubly indented itemized list\cr
\\"\hangindent="\<dimen>&hanging indentation for paragraph\cr
\\"\hangafter="\<number>&start hanging indent after line $n$.\cr
   &\omit\hfil If $n<0$, indent first $|n|$ lines.\cr
\\"\parshape="\<number>&general paragraph shaping macro\cr
}


\section{Headers, Footers, and Page Numbers}

\halign{#\hfil\qquad&#\hfil\cr
\\"\nopagenumbers"&turn off page numbering\cr
\\"\pageno"&current page number. To get roman nums,\cr
   &\omit\hfil   set \\"\pageno="\<negative number>\cr
\\"\folio"&current page number, roman num if ${}<0$\cr
\\"\footline"&material to put at foot of page\cr
\\"\headline"&material to put at top of page. To leave\cr
   &\quad  space, set \\"\voffset=2\baselineskip", make\cr
   &\quad  room with \\"\advance\vsize by-\voffset".\cr
}

\section{Macro Definitions}

\paralign{#\hfil\qquad&#\hfil\cr
\\"\def\cs{"\<replacement text>\\"}"
   &define the macro \\"\cs"\hfil\cr
\\"\def\cs#1"${}\cdots{}\hbox{\tt\#}n$\\"{"\<repl.\ text>\\"}"
   &macro with parameters\cr
\multispan2\ignorespaces
\\"\let\cs="\<token>\hfil give \\"\cs" token's current meaning\cr
}


\begingroup
\parskip=0pt
\begintext
{\bf Advanced Macro Definition Commands}
\endtext

\paralign{#\hfil\qquad&#\hfil\cr
\\"\long\def"&macro whose args may include \\"\par"\cr
\\"\outer\def"&macro not allowed inside definitions\cr
\\"\global\def" or \\"\gdef"&definition that transcends grouping\cr
\\"\edef"&expand while defining macro\cr
\\"\xdef" or \\"\global\edef"&global version of \\"\edef"\cr
\\"\noexpand"\<token>&do not expand token\cr
\\"\expandafter"\<token>&expand item after token first\cr
\multispan2\ignorespaces
\\"\futurelet\cs"\<tok$_1$>\<tok$_2$>
   \hfil equals \\"\let\cs="\<tok$_2$>\<tok$_1$>\<tok$_2$>\cr
\\"\csname"\dots\\"\endcsname"&create a control sequence name\cr
\\"\string\cs"&list characters in name, \\"\"\ \\"c"\ \\"s"\cr
\\"\number"\<number>&list of characters in number\cr
\\"\the"\<internal quantity>&list of tokens giving value of quantity\cr
}
\endgroup

\section{Conditionals}

\begingroup
\parskip=0pt
\begintext
The general format of a conditional is
\endtext
\tabskip=\centering
\halign to\hsize{\hfil#\hfil\cr
\\"\if"\<condition>\<true text>\\"\else"\<false text>\\"\fi"\cr
}
\paralign{#\hfil\qquad&#\hfil\cr
\\"\ifnum"\<num$_1$>\<relation>\<num$_2$>&compare two integers\cr
\\"\ifdim"\<dimen$_1$>\<relation>\<dimen$_2$>&compare two dimensions\cr
\\"\ifodd"\<num>&test for an odd integer\cr
\\"\ifmmode"&test for math mode\cr
\\"\if"\<token$_1$>\<token$_2$>&test if character codes agree\cr
\\"\ifx"\<token$_1$>\<token$_2$>&test if tokens agree\cr
\\"\ifdim"\<dim$_1$>\<dim$_2$>&test if dimensions agree\cr
\\"\ifeof"\<number>&test for end of file\cr
\\"\iftrue", \\"\iffalse"&always true, always false\cr
\\"\ifcase"\<number>\<text$_0$>\\"\or"\<text$_1$>\\"\or"$\cdots$\hidewidth\cr
   \qquad \\"\or"\<text$_n$>\\"\else"\<text>\\"\fi"&choose text by \<number>\cr
\multispan2\ignorespaces
\\"\loop" $\alpha$ \\"\if"\dots $\beta$ \\"\repeat"\qquad
   loop $\alpha\beta\alpha\cdots\alpha$ until \\"\if" is false\cr
}
\endgroup
\vskip-.5\baselineskip
\paralign{#\hfil\qquad&#\hfil\cr
\\"\newif\ifblob"&create a new  conditional called \\"\ifblob"\cr
\\"\blobtrue", \\"\blobfalse"&set conditional \\"\ifblob" true, false\cr
%\\"\blobtrue"&set the conditional \\"\ifblob" to be true\cr
%\\"\blobfalse"&set the conditional \\"\ifblob" to be false\cr
}

\section{Dimensions, Spacing, and Glue}
\begintext
Dimensions are specified as  \<number>\<unit of measure>.
\par
Glue is specified as \<dimen> {\tt plus}\<dimen> {\tt minus}\<dimen>.
\endtext

\vbox{%
\offinterlineskip
\halign to\hsize{%
\strut#\hfil\tabskip=1em&\tt#\hfil\tabskip=0pt plus1fil&\vrule#
&#\hfil\tabskip=1em&\tt#\hfil\tabskip=0pt plus1fil&\vrule#
&#\hfil\tabskip=1em&\tt#\hfil\tabskip=0pt plus1fil&\vrule#
&#\hfil\tabskip=1em&\tt#\hfil\tabskip=0pt\cr
point&pt&&pica&pc&&inch&in&&centimeter&cm\cr
m width&em&&x height&ex&&math unit&mu&&millimeter&mm\cr
\omit&& height 2pt&&& height 2pt&&& height 2pt\cr
\multispan2 1 pc = 12 pt\hfil&&
\multispan2 1 in = 72.72 pt\hfil&&
\multispan2 2.54 cm = 1 in\hfil&&
\multispan2 18 mu = 1 em\hfil\tabskip=0pt\cr
}}



\paralign{%
#\quad\hfil&#\hfil\cr
Horizontal Spacing:&
\\"\quad" (skip 1em)\quad
\\"\qquad"\cr
Horizontal Spacing (Text):&
   \\"\thinspace"\quad
   \\"\enspace"\quad
   \\"\enskip"
\cr
\multispan1\hfil
   \\"\hskip"\<glue>\quad
   \\"\hfil"\quad
   \\"\hfill"\quad
   \\"\hfilneg"\hidewidth
\cr
Horizontal Spacing (Math):&
   thin space \\"\," \quad
   medium space \\"\>"
\cr
\multispan1\hfil
   thick space \\"\;"\quad
   neg.\ thin space \\"\!"\quad
   \\"\mskip"\<muglue>\hidewidth
\cr
\noalign{\vskip1\parskip}
Vertical Spacing:&
   \\"\vskip"\<glue>\quad
   \\"\vfil"\quad
   \\"\vfill"
\cr
\noalign{\vbox{
\halign{\qquad#\quad\hfil&#\hfil\cr
\\"\strut"&box w/ ht and depth of ``('', zero width\cr
\\"\phantom{"\<text>\\"}"&invisible box with dim of \<text>\cr
\\"\vphantom{"\<text>\\"}"&box w/ ht \& depth of \<text>, zero width\cr
\\"\hphantom{"\<text>\\"}"&box w/ width of \<text>, zero ht \& depth\cr
\\"\smash{"\<text>\\"}"&typeset \<text>, set ht \& depth to zero\cr
}
\halign{\qquad#\quad\hfil&#\hfil\cr
\\"\raise"\<dimen>\\"\hbox{"\<text>\\"}"&raise box up\cr
\\"\lower"\<dimen>\\"\hbox{"\<text>\\"}"&lower  box down\cr
\\"\moveleft"\<dimen>\\"\vbox{"\<text>\\"}"&move  box left\cr
\\"\moveright"\<dimen>\\"\vbox{"\<text>\\"}"&move  box right\cr
}
}}
Skip Space Between Lines:&
   \\"\smallskip"\quad\\"\medskip"\quad\\"\bigskip"\cr
   \qquad encourage a break\qquad
     \\"\smallbreak"\quad\\"\medbreak"\quad\\"\bigbreak"\hidewidth\cr
   \qquad break if no room&\\"\filbreak"\cr
Set Line Spacing:&
   \\"\baselineskip =" \<glue>\cr
   \qquad single space&\\"\baselineskip = 12pt"\cr
   \qquad 1 1/2 space&\\"\baselineskip = 18pt"\cr
   \qquad double space&\\"\baselineskip = 24pt"
\cr
Increase Line Spacing&
   \\"\openup"\<dimen>\cr
   \qquad use \\"\jot"'s&\\"1\jot = 3pt"
\cr
Allow Unjustified Lines&\\"\raggedright"\cr
Allow Unjustified Pages&\\"\raggedbottom"\cr
}

\section{Braces and Matrices}

\halign{#\hfil\qquad&#\hfil\cr
\\"\matrix"&rectangular array of entries\cr
\\"\pmatrix"&matrix with parentheses\cr
\\"\bordermatrix"&matrix with labels on top and left\cr
\\"\overbrace"&overbrace, may be superscripted\cr
\\"\underbrace"&underbrace, may be subscripted\cr
}

\begintext
For small matrices in text, use the following constructions:
\endtext
\halign{\qquad#\hfil&\qquad$#$\hfil\cr
\\"{a\,b \choose c\,d}"&a\,b \choose c\,d\cr
\\"\left( {a\atop c} {b\atop d} \right)"
         &\left( {a\atop c}  {b\atop d} \right)\cr
}

\section{Displayed Equations}

\halign{#\hfil\qquad&#\hfil\cr
\\"\eqno"&equation number at right\cr
\\"\leqno"&equation number at left\cr
\\"\eqalign"&display several aligned equations\cr
\\"\eqalignno"&display aligned equations numbered at right\cr
\\"\leqalignno"&display aligned equations numbered at left\cr
\\"\displaylines"&display several equations, centered\cr
\\"\cases"&case by case definitions\cr
\\"\noalign"&to insert space between lines in displays,\cr
   &\enspace use \\"\noalign{\vskip"\<glue>\\"}" after any \\"\cr"\cr
\\"\openup"\<dimen>&add space between all lines in a display\cr
}

\copyrightnotice

\bye

% *************** END OF MATERIAL THAT FITS INTO SIX COLUMNS ****************

% Additional material that didn't fit into six columns, including an expanded
% version of the "Conditionals" and "Every Time Insertions" sections.

\section{Conditionals}

\begintext
The general format of a conditional is
\endtext
\begingroup
\parskip=0pt
\tabskip=\centering
\halign to\hsize{\hfil#\hfil\cr
\\"\if"\<condition>\<true text>\\"\else"\<false text>\\"\fi"\cr
}
\paralign{#\hfil\qquad&#\hfil\cr
\\"\ifnum"\<num$_1$>\<relation>\<num$_2$>&compare two integers\cr
\\"\ifdimen"\<dim$_1$>\<relation>\<dim$_2$>&compare two dimensions\cr
\\"\ifodd"\<num>&test for an odd integer\cr
\\"\ifvmode"&test for vertical mode\cr
\\"\ifhmode"&test for horizontal mode\cr
\\"\ifmmode"&test for math mode\cr
\\"\ifinner"&test for an internal mode\cr
\\"\if"\<token$_1$>\<token$_2$>&test if character codes agree\cr
\\"\ifcat"\<token$_1$>\<token$_2$>&test if category codes agree\cr
\\"\ifx"\<token$_1$>\<token$_2$>&test if tokens agree\cr
\\"\ifvoid"\<number>&test if box register is empty\cr
\\"\ifeof"\<number>&test for end of file\cr
\\"\iftrue", \\"\iffalse"&always true, always false\cr
\\"\ifcase"\<number>\<text$_0$>\\"\or"\<text$_1$>\\"\or"$\cdots$\hidewidth\cr
   \qquad \\"\or"\<text$_n$>\\"\else"\<text>\\"\fi"&choose text by \<number>\cr
\multispan2\ignorespaces
\\"\loop" $\alpha$ \\"\if"\dots $\beta$ \\"\repeat"\qquad
   loop $\alpha\beta\alpha\cdots\alpha$ until \\"\if" is false\cr
}
\endgroup
\paralign{#\hfil\qquad&#\hfil\cr
\\"\newif\ifblob"&create a new  conditional called \\"\ifblob"\cr
\\"\blobtrue"&set the conditional \\"\ifblob" to be true\cr
\\"\blobfalse"&set the conditional \\"\ifblob" to be false\cr
}

\section{Advanced Control Sequences}

\halign{#\hfil\qquad&#\hfil\cr
\\"\mathchoice{"\<f$_1$>\\"}",\\"{"\<f$_2$>\\"}",\\"{"\<f$_3$>\\"}",
   \\"{"\<f$_4$>\\"}"\hidewidth\cr
   &choose formula based on current style\cr
\\"\mathpalette\f{xyz}"&expands to \\"\f{"\<style>\\"}{xyz}"\cr
}

\section{Every Time Insertions}
\halign{#\hfil\qquad&#\hfil\cr
\\"\everypar"&insert whenever a paragraph begins\cr
\\"\everymath"&insert whenever math in text begins\cr
\\"\everydisplay"&insert whenever displayed math begins\cr
\\"\everyhbox"&insert whenever an hbox begins\cr
\\"\everyvbox"&insert whenever a vbox begins\cr
\\"\everycr"&insert after every \\"\cr"\cr
}


\bye
